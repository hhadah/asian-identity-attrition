\begin{table}[H]

\caption{Relationship Between Bias and Interethnic Marriages \label{regtab-logit-02}}
\centering
\begin{threeparttable}
\begin{tabular}[t]{lccc}
\toprule
\multicolumn{2}{c}{ } & \multicolumn{1}{c}{Asian Men} & \multicolumn{1}{c}{Asian Women} \\
\cmidrule(l{3pt}r{3pt}){3-3} \cmidrule(l{3pt}r{3pt}){4-4}
  & \specialcell{(1) \\ Interethnic} & \specialcell{(2) \\ Interethnic} & \specialcell{(3) \\ Interethnic}\\
\midrule
Bias & 0.11*** & -0.20*** & -0.03\\
 & (0.03) & (0.03) & (0.03)\\
College Graduate: Wife & 0.03*** & 0.03*** & 0.03***\\
 & (0.00) & (0.00) & \vphantom{1} (0.00)\\
College Graduate: Husband & 0.01* & 0.00 & 0.00\\
 & (0.00) & (0.00) & (0.00)\\
\midrule
Observations & 146,125 & 108,152 & 125,085\\
Year $\times$ Region FE & X & X & X\\
\bottomrule
\multicolumn{4}{l}{\rule{0pt}{1em}* p $<$ 0.1, ** p $<$ 0.05, *** p $<$ 0.01}\\
\end{tabular}
\begin{tablenotes}
\small
\item[1] \footnotesize{This is the result to estimating (\ref{eq:inter-interethnic}) as a
                      linear probability model.}
\item[2] \footnotesize{I include controls for partners' sex, age, education, 
                      and years since immigrating to the United States.
                      Standard errors are clustered on the household level.}
\item[3] \footnotesize{Data source is the 2004-2020 Current Population Survey Data.}
\end{tablenotes}
\end{threeparttable}
\end{table}
