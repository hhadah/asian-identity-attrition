\begin{table}[H]

\caption{Asian Self-identification by Generation \label{tab:hispbygen}}
\centering
\begin{threeparttable}
\begin{tabular}[t]{>{}lcccc}
\toprule
  & \specialcell{Self-identify \\ as Asian} & \specialcell{Self-identify as \\ non-Asian} & \specialcell{\% Self-identify \\ as Asian} & \specialcell{\% Self-identify \\ as non-Asian}\\
\midrule
\textbf{1st Gen.} & 38,198 & 1,835 & 0.95 & 0.05\\
\textbf{2nd Gen.} & 143,066 & 56,228 & 0.72 & 0.28\\
\hspace{1em}\textbf{Asian on:} &  &  &  \vphantom{1} & \\
\hspace{1em}\hspace{1em}\textbf{Both Sides} & 116,593 & 4,272 & 0.96 & 0.04\\
\hspace{1em}\hspace{1em}\textbf{One Side} & 26,473 & 51,956 & 0.34 & 0.66\\
\addlinespace
\textbf{3rd Gen.} & 25,797 & 53,280 & 0.33 & 0.67\\
\hspace{1em}\textbf{Asian on:} &  &  &  & \\
\hspace{1em}\hspace{1em}\textbf{Both Sides} & 11,204 & 547 & 0.95 & 0.05\\
\hspace{1em}\hspace{1em}\textbf{One Side} & 6,188 & 19,225 & 0.24 & 0.76\\
\bottomrule
\end{tabular}
\begin{tablenotes}
\item[1] The samples include children ages 17 and below who live in intact families. First-generation Asian immigrant children that were born in a Asian country. Native-born second-generation Asian immigrant children with at least one parent born in a Asian country. Finally, native-born third-generation Asian immigrant children with native-born parents and at least one grandparent born in a Asian country.
\item[2] Data source is the 2004-2021 Current Population Survey.
\end{tablenotes}
\end{threeparttable}
\end{table}
