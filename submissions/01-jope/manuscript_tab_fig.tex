
%%%%%%%%%%%%%%%%%%%%%%%%%%%%%%%%%%%%
% Tables
%%%%%%%%%%%%%%%%%%%%%%%%%%%%%%%%%%%%
\begin{landscape}
\begin{table}[H]
\centering
\caption{CPS Summary Statistics with Skin IAT Scores \label{tab:sumstat1}}
\centering
\begin{threeparttable}
\resizebox{\ifdim\width>\linewidth\linewidth\else\width\fi}{!}{
\begin{tabular}[t]{lcccc}
\toprule
\multicolumn{1}{c}{ } & \multicolumn{1}{c}{\textbf{Overall}} & \multicolumn{3}{c}{\textbf{By Generation}} \\
\cmidrule(l{3pt}r{3pt}){2-2} \cmidrule(l{3pt}r{3pt}){3-5}
\textbf{Characteristic} & \makecell[c]{\textbf{All Sample} \\N = 318,404} & \makecell[c]{\textbf{First} \\N=40,033} & \makecell[c]{\textbf{Second} \\N=199,294} & \makecell[c]{\textbf{Third} \\N=79,077}\\
\midrule
Female & 0.49 & 0.53 & 0.49 & 0.49\\
Asian & 0.65 & 0.96 & 0.73 & 0.31\\
Age & 8.4 (5.1) & 10.9 (4.5) & 8.3 (5.1) & 7.7 (5.0)\\
College Graduate:\ \ 	 Father & 0.52 & 0.59 & 0.52 & 0.50\\
College Graduate:\ \ 	 Mother & 0.52 & 0.56 & 0.51 & 0.52\\
Total Family Income\ \ 	 (1999 dollars) & 87,031 (84,797) & 75,815 (74,489) & 88,295 (88,411) & 89,436 (80,051)\\
\bottomrule
\end{tabular}}
\begin{tablenotes}
\item[1] The samples include children ages 17 and below who live in intact families. First-generation Asian immigrant children that were born in a Asian country. Native-born second-generation Asian immigrant children with at least one parent born in a Asian country. Finally, native-born third generation Asian immigrant children with native-born parents and at least one grand parent born in a Asian country.
\item[2] Data source is the 2004-2021 Current Population Survey.
\end{tablenotes}
\end{threeparttable}
\end{table}

\end{landscape}

\newpage
\pagebreak

\begin{table}[H]
\centering\centering
\caption{Asian Self-identification by Generation \label{tab:hispbygen}}
\centering
\begin{threeparttable}
\begin{tabular}[t]{>{}lcccc}
\toprule
  & \specialcell{Self-identify \\ as Asian} & \specialcell{Self-identify as \\ non-Asian} & \specialcell{\% Self-identify \\ as Asian} & \specialcell{\% Self-identify \\ as non-Asian}\\
\midrule
\textbf{1st Gen.} & 14,811 & 688 & 0.96 & 0.04\\
\textbf{2nd Gen.} & 58,756 & 21,381 & 0.73 & 0.27\\
\hspace{1em}\textbf{Asian on:} &  &  &  \vphantom{1} & \\
\hspace{1em}\hspace{1em}\textbf{Both Sides} & 49,118 & 1,717 & 0.97 & 0.03\\
\hspace{1em}\hspace{1em}\textbf{One Side} & 9,638 & 19,664 & 0.33 & 0.67\\
\addlinespace
\textbf{3rd Gen.} & 10,394 & 23,048 & 0.31 & 0.69\\
\hspace{1em}\textbf{Asian on:} &  &  &  & \\
\hspace{1em}\hspace{1em}\textbf{Both Sides} & 5,428 & 316 & 0.94 & 0.06\\
\hspace{1em}\hspace{1em}\textbf{One Side} & 3,030 & 9,213 & 0.25 & 0.75\\
\bottomrule
\end{tabular}
\begin{tablenotes}
\item[1] The samples include children ages 17 and below who live in intact families. First-generation Asian immigrant children that were born in a Asian country. Native-born second-generation Asian immigrant children with at least one parent born in a Asian country. Finally, native-born third-generation Asian immigrant children with native-born parents and at least one grandparent born in a Asian country.
\item[2] Data source is the 2004-2021 Current Population Survey.
\end{tablenotes}
\end{threeparttable}
\end{table}


\begin{center}
\begin{figure}[H]
\caption{Bias and Self-reported Asian Identity in the Least and Most Biased Places}

% first
\begin{subfigure}{.9\textwidth}
\caption{Skin Tone Implicit Association Bias Over Time}
\centering
\includegraphics[width=.9\linewidth]{Bias_twostates.png} 
\label{fig:skiniat}
\end{subfigure}
% Second
\begin{subfigure}{.9\textwidth}
\caption{Self-reported Asian Identity Over Time}
\centering
\includegraphics[width=.9\linewidth]{Bias_twostates-asian.png} 
\label{fig:Asian-twostates}
\end{subfigure}
\caption*{\footnotesize{These two panels show the trends in implicit bias (panel a) and self-reported Asian identity among Asian immigrants (panel b) of the least and most biased places in the data. The District of Colombia is the least biased geographical area, and North Dakota is the most biased. The bias units are in standard deviations. Self-reported Asian identity is among first, second, and third-generation Asian immigrants aged 17 and younger still living in intact families.\\
Bias data is from the 2004-2021 Harvard's Project Implicit Association Test scores, American National Election Studies (ANES), and state-level hate crimes against Asians. Identity data is from the 2004-2021 Current Population Survey (CPS).}}
\end{figure}
\end{center}


\newpage
\pagebreak

\begin{center}
\begin{figure}[H]
\caption{Diagram of the Three Different Generations of Asian Immigrants.}
\includegraphics[width=\textwidth, height=9cm]{diag.png} 
\label{fig:diag}
\end{figure}
\hfill%
\end{center}

\pagebreak
\newpage

\begin{center}
\begin{figure}[H]
\caption{Maps of State-level Implicit Association Test Bias Over Time Measure with Census Division Regional Boundaries}
\label{fig:skiniat-maps}
% first
\begin{subfigure}{.45\textwidth}
\caption{State-level Bias in 2004}
\centering
\includegraphics[width=0.9\linewidth]{2004skinmap.png} 
\label{fig:skiniat-map-2004}
\end{subfigure}
\hfill%
% Second
\begin{subfigure}{.45\textwidth}
\caption{State-level Bias in 2008}
\centering
\includegraphics[width=0.9\linewidth]{2008skinmap.png} 
\label{fig:skiniat-map-2006}
\end{subfigure}
\hfill%
% third
\begin{subfigure}{.45\textwidth}
\caption{State-level Bias in 2012}
\centering
\includegraphics[width=0.9\linewidth]{2012skinmap.png} 
\label{fig:skiniat-map-2008}
\end{subfigure}
\hfill%
% fourth
\begin{subfigure}{.45\textwidth}
\caption{State-level Bias in 2016}
\centering
\includegraphics[width=0.9\linewidth]{2016skinmap.png} 
\label{fig:skiniat-map-2010}
\end{subfigure}

\caption*{\footnotesize{This figure shows the state-level bias index in different years in the sample. The bias units are in standard deviations and ranges from low to high bias. Bias index is constructed following \textcite{lubotskyInterpretationRegressionsMultiple2006}. The data is from the 2004-2021 Harvard's Project Implicit Association Test scores, American National Election Studies (ANES), and state-level hate crimes against Asians. Each panel presents state-level bias during a certain year. The boundaries in black represent the different Census divisions in the United States. Notice how there is a variation across states within a region.}}
\end{figure}
\end{center}

\newpage
\pagebreak

\begin{center}
\begin{figure}[H]
\caption{Maps of State-level Bias 2004-2021 Measure with Census Division Regional Boundaries}
\includegraphics[width=\textwidth]{Average_Skinmap.png} 
\label{fig:iat-map-all}
\caption*{\footnotesize{This figure shows the state-level bias index in the sample from 2004 to 2021. The bias units are in standard deviations and ranges from low to high bias. Bias index is constructed following \textcite{lubotskyInterpretationRegressionsMultiple2006}. The data is from the 2004-2021 Harvard's Project Implicit Association Test scores, American National Election Studies (ANES), and state-level hate crimes against Asians. The boundaries in black represent the different Census divisions in the United States. Notice how there is a variation across states within a region.}}
\end{figure}
\end{center}

\pagebreak
\newpage

\begin{center}
\begin{figure}[!htb]
\centering
\caption{Relationship Between Self-Reported Asian Identity and Bias: By Generation}
\label{plot01-regression-gen}
%First graph
\begin{subfigure}{.48\textwidth}
\caption{All Generations}
\centering
\includegraphics[width=.9\linewidth]{skin-iat-regression-all-gens.png}
\end{subfigure}
\centering
%Second graph
\begin{subfigure}{.48\textwidth}
\caption{First-Generation}
\centering
\includegraphics[width=.9\linewidth]{skin-iat-regression-first-gen.png}
\end{subfigure}
%Third Graph
\begin{subfigure}{.48\textwidth}
\caption{Second-Generation}
\centering
\includegraphics[width=.9\linewidth]{skin-iat-regression-second-gen.png}
\end{subfigure}
%Fourth Graph
\begin{subfigure}{.48\textwidth}
\caption{Third-Generation}
\centering
\includegraphics[width=.9\linewidth]{skin-iat-regression-third-gen.png}
\end{subfigure}
\caption*{\footnotesize{I show four panels of estimating equation (\ref{eq:identity_reg_bias}). I include region $\times$ year fixed effects with controls for sex, quartic age, and parental education. The dependent variable is self-reported Asian identity and the independent variable is state-level bias. Each panel is the results from the same regression but on different samples that are divided by generation. Standard errors are clustered on the state level. The samples include first-, second-, and third-generation Asian children ages 17 and below who live in intact families. First-generation Asian immigrants are children that were born in a Asian country. Native-born second-generation Asian immigrants are children with at least one parent born in a Asian country. Finally, native-born third-generation Asian immigrants are children with native-born parents and at least one grandparent born in a Asian country.}}
\end{figure}
\end{center}

\pagebreak
\newpage

\begin{center}
\begin{figure}[!htb]
\centering
\caption{Relationship Between Self-Reported Asian Identity and Bias: By Parental Types}
\label{plot01-regression-byparent}
%First graph
\begin{subfigure}{.48\textwidth}
\caption{Second-Generation (All Parental Types)}
\centering
\includegraphics[width=.9\linewidth]{by-parents-regs-all.png}
\end{subfigure}
\centering
%Second graph
\begin{subfigure}{.48\textwidth}
\caption{Asian Fathers-Asian Mothers}
\centering
\includegraphics[width=.9\linewidth]{by-parents-regs-hh.png}
\end{subfigure}
%Third Graph
\begin{subfigure}{.48\textwidth}
\caption{Asian Fathers-White Mothers}
\centering
\includegraphics[width=.9\linewidth]{by-parents-regs-hw.png}
\end{subfigure}
%Fourth Graph
\begin{subfigure}{.48\textwidth}
\caption{White Fathers-Asian Mothers}
\centering
\includegraphics[width=.9\linewidth]{by-parents-regs-wh.png}
\end{subfigure}
\caption*{\footnotesize{I show four panels of estimating equation (\ref{eq:identity_reg_bias}). I include region $\times$ year fixed effects with controls for sex, quartic age, and parental education. The dependent variable is self-reported Asian identity and the independent variable is state-level bias. Each panel results from the same regression but on different samples divided by parental types. Standard errors are clustered on the state level. The samples include second-generation Asian children ages 17 and below who live in intact families. Native-born second-generation Asian immigrant children with at least one parent born in a Spanish-speaking country.}}
\end{figure}
\end{center}

\pagebreak
\newpage

\begin{table}[H]
\centering\centering
\caption{Relationship Between Bias and Self-Reported Asian identity Among Third-Generation Asian Immigrants: By Grandparental Type \label{regtab-bygrandparents}}
\centering
\resizebox{\ifdim\width>\linewidth\linewidth\else\width\fi}{!}{
\begin{threeparttable}
\begin{tabular}[t]{lcccc}
\toprule
\multicolumn{1}{c}{ } & \multicolumn{4}{c}{Number of Asian Grandparents} \\
\cmidrule(l{3pt}r{3pt}){2-5}
  & \specialcell{(1) \\ One} & \specialcell{(2) \\ Two} & \specialcell{(3) \\ Three} & \specialcell{(4) \\ Four}\\
\midrule
Bias & -0.01 & -0.09 & -0.69** & -0.11\\
 & (0.04) & (0.08) & (0.32) & (0.06)\\
Female & -0.01 & -0.01 & -0.04 & -0.03**\\
 & (0.01) & (0.02) & (0.06) & (0.01)\\
College Graduate: Mother & 0.01 & 0.07** & 0.08 & 0.00\\
 & (0.01) & (0.03) & (0.09) & (0.03)\\
College Graduate: Father & -0.04*** & 0.00 & -0.07 & 0.00\\
 & (0.01) & (0.04) & (0.08) & (0.01)\\
\midrule
Observations & 14,453 & 12,678 & 567 & 5,744\\
Year $\times$ Region FE & X & X & X & X\\
\bottomrule
\multicolumn{5}{l}{\rule{0pt}{1em}* p $<$ 0.1, ** p $<$ 0.05, *** p $<$ 0.01}\\
\end{tabular}
\begin{tablenotes}
\small
\item[1] \footnotesize{Each column is an estimation of equation (\ref{eq:identity_reg_bias}) restricted to third-generation Asian immigrants by 
                      number of Asian grandparents with region × year fixed effects. 
                      I include controls for sex, quartic age, fraction of Asians in a state, and parental education.
                      Standard errors are clustered on the state level.}
\item[2] \footnotesize{The samples include third-generation Asian children ages 17 and below who live in intact families. 
                      Native-born third-generation Asian 
                      immigrant children with at least one grandparent born in a Asian 
                      country.}
\item[3] \footnotesize{Data source is the 2004-2021 Current Population Survey.}
\end{tablenotes}
\end{threeparttable}}
\end{table}


\pagebreak
\newpage

\begin{table}[H]

\caption{Relationship Between Bias and Interethnic Marriages \label{regtab-logit-02}}
\centering
\begin{threeparttable}
\begin{tabular}[t]{lccc}
\toprule
\multicolumn{2}{c}{ } & \multicolumn{1}{c}{Asian Men} & \multicolumn{1}{c}{Asian Women} \\
\cmidrule(l{3pt}r{3pt}){3-3} \cmidrule(l{3pt}r{3pt}){4-4}
  & \specialcell{(1) \\ Interethnic} & \specialcell{(2) \\ Interethnic} & \specialcell{(3) \\ Interethnic}\\
\midrule
Bias & 0.11*** & -0.20*** & -0.03\\
 & (0.03) & (0.03) & (0.03)\\
College Graduate: Wife & 0.03*** & 0.03*** & 0.03***\\
 & (0.00) & (0.00) & \vphantom{1} (0.00)\\
College Graduate: Husband & 0.01* & 0.00 & 0.00\\
 & (0.00) & (0.00) & (0.00)\\
\midrule
Observations & 146,125 & 108,152 & 125,085\\
Year $\times$ Region FE & X & X & X\\
\bottomrule
\multicolumn{4}{l}{\rule{0pt}{1em}* p $<$ 0.1, ** p $<$ 0.05, *** p $<$ 0.01}\\
\end{tabular}
\begin{tablenotes}
\small
\item[1] \footnotesize{This is the result to estimating (\ref{eq:inter-interethnic}) as a
                      linear probability model.}
\item[2] \footnotesize{I include controls for partners' sex, age, education, 
                      and years since immigrating to the United States.
                      Standard errors are clustered on the household level.}
\item[3] \footnotesize{Data source is the 2004-2020 Current Population Survey Data.}
\end{tablenotes}
\end{threeparttable}
\end{table}


\pagebreak
\newpage

\begin{table}[H]

\caption{Relationship Between Bias and Migration \label{regtab-mig-01}}
\centering
\resizebox{\linewidth}{!}{
\begin{threeparttable}
\begin{tabular}[t]{lccc}
\toprule
  & \specialcell{(1) \\ Migrated from \\ Birth Place} & \specialcell{(2) \\ Migrated from \\ Birth Place} & \specialcell{(3) \\ $Bias_{ist} - Bias_{ilb}$}\\
\midrule
$Bias_{st}$ & 1.05*** &  & \\
 & (0.18) &  & \\
$Bias_{lb}$ &  & 0.15 & \\
 &  & (0.59) & \\
Asian &  &  & 0.01**\\
 &  &  & (0.01)\\
Female & 0.00** & -0.01*** & 0.00\\
 & (0.00) & (0.00) & (0.00)\\
College Graduate: Mother & 0.01*** & 0.00 & -0.01**\\
 & (0.00) & (0.01) & (0.00)\\
College Graduate: Father & -0.03*** & -0.03*** & 0.01**\\
 & (0.01) & (0.01) & (0.00)\\
\midrule
Observations & 169,206 & 86,686 & 10,410\\
Mean & 0.15 & 0.15 & -0.06\\
Year $\times$ Region FE & X &  & \\
Birthyear $\times$ Birth Region FE &  & X & \\
\bottomrule
\multicolumn{4}{l}{\rule{0pt}{1em}* p $<$ 0.1, ** p $<$ 0.05, *** p $<$ 0.01}\\
\end{tabular}
\begin{tablenotes}
\small
\item[1] \footnotesize{Each column is an estimation of equations (\ref{eq:migration-3}) in column (1), 
                      (\ref{eq:migration-4}) in column (2), and
                      (\ref{eq:migration-5}) in column (3).}
\item[2] \footnotesize{Column (1) is a regression where the left hand side variable is 
                      a dummy variable that is equal to one if a person migrated from the state
                      were born in and the right hand side variable is bias the year of survey.
                      Column (2) is a regression where the left hand side variable is 
                      a dummy variable that is equal to one if a person migrated from the state
                      were born in and the right hand side variable is bias the year of birth in the state of birth.
                      Column (3) is a regression where the left hand side variable is 
                      the difference between state-level bias during the year of the survey in the current state the 
                      respondent is living in, and state-level bias during the year of birth in the state of birth 
                      and the right hand side variable is self-reported Asian identity. This regression captures
                      the selection of those that self-reported Asian identity into states with different levels of bias.
                      I include controls for sex, quartic age, parental education, fraction of Asians in a state, and region × year fixed effects.
                      Standard errors are clustered on the state level.}
\item[3] \footnotesize{The samples include children ages 17 and below who live in intact families. 
                      Native-born second-generation Asian immigrant children with both
                      parents born in a Asian country. The sample in the column (3) regression is further restricted to only those that migrated from their birth state.}
\item[4] \footnotesize{Data source is the 2004-2021 Census Data.}
\end{tablenotes}
\end{threeparttable}}
\end{table}


\pagebreak
\newpage

\begin{table}[H]
\centering\centering
\caption{Main Effect of Proxy on Second-Generation's Asian Self-identification \label{tab:hispbyproxy}}
\centering
\fontsize{12}{14}\selectfont
\begin{tabular}[c]{>{}lllll}
\toprule
Parents Type & All & Asian-Asian & Asian-White & White-Asian\\
\midrule
\textbf{Proxy:} &  &  &  & \\
\hspace{1em}\textbf{Mother} & 0.72 & 0.97 & 0.37 & 0.3\\
\hspace{1em}\textbf{Father} & 0.72 & 0.97 & 0.39 & 0.29\\
\hspace{1em}\textbf{Self} & 0.87 & 0.97 & 0.23 & 0.31\\
\hspace{1em}\textbf{Others} & 0.88 & 0.96 & 0.6 & 0.54\\
\bottomrule
\end{tabular}
\end{table}

\pagebreak
\newpage

\clearpage
