%%%%%%%%%%%%%%%%%%%%%%%%%%%%%%%%%%%
% Main Text
%%%%%%%%%%%%%%%%%%%%%%%%%%%%%%%%%%%

\section{Introduction}\label{sec:intro}

Asian Americans represent one of the fastest-growing racial groups in the United States, yet their experiences with discrimination and identity formation remain underexplored in economic research.\footnote{The 2020 Census counted more than 20 million Asian Americans---6.4 percent of the population---nearly double the number counted two decades earlier \autocite{floodsarahIntegratedPublicUse2021a}. The Asian American population numbers are based on the author's calculations from the Current Population Survey and US Census data.} Unlike other minority groups, Asian Americans occupy a distinctive position in America's racial environment, simultaneously experiencing discrimination while being characterized through the ``model minority'' stereotype. This dual status creates complex incentives around racial identity choices that fundamentally differ from other groups' experiences.

An extensive literature has documented Asian-White gaps in various outcomes \autocite{chiswick1983analysis, duleep2012economic, hilger2016upward, arabsheibani2010asian}, yet the role of identity selection in shaping these disparities remains understudied. The challenge lies in defining and measuring racial identity, particularly when individuals possess agency in how they racially self-identify. To the extent that reporting Asian racial identity represents a strategic choice influenced by local discrimination, measured gaps may systematically vary across geographic contexts. For example, if high ability individuals in high-discrimination areas choose to identify as White, then conventional analyses may overestimate Asian-White gaps in those states, leading to misleading conclusions about Asian American integration and success.

Various contextual factors, including anti-Asian sentiment and stereotypes, can influence how individuals choose their racial identity choices. In this paper, I examine the determinants of Asian racial identity and analyze how Asian Americans strategically select between Asian and White racial identities. Specifically, I investigate how anti-Asian bias shapes decisions to racially identify, or not, as Asian American. This question is critical for several reasons. First, if individuals respond to prejudice by avoiding Asian racial identification, conventional analyses of racial gaps may systematically overestimate disparities in the most prejudiced areas. Second, identity choices may influence measured labor market trajectories among racial groups, potentially making Asian American integration appear more successful than reality suggests. Third, strategic identity reporting affects the enumeration of Asian American populations, with implications for political representation and resource distribution.

I explore how individual characteristics and societal attitudes toward Asian Americans influence racial identity reporting. I utilize identity and ancestry data from the Current Population Survey (CPS) combined with measures of anti-Asian bias derived from Harvard's Project Implicit Association Test (IAT), the American National Election Studies (ANES), and hate crimes targeting Asian Americans.\footnote{The IAT data comes from Harvard's Project Implicit \autocite{greenwaldMeasuringIndividualDifferences1998}. Implicit bias measures have gained prominence in economics, with IAT scores correlating with economic outcomes \autocite{chettyRaceEconomicOpportunity2020,gloverDiscriminationSelfFulfillingProphecy2017}, voting patterns \autocite{friesePredictingVotingBehavior2007}, and health disparities \autocite{leitnerRacialBiasAssociated2016}.} I ground my analysis in a theoretical framework extending \textcite{akerlofEconomicsIdentity2000}, explicitly modeling how external prejudice creates differential utility from identity choices and establishing conditions under which individuals strategically modify their racial self-presentation.

Measuring identity choices outside of a laboratory is challenging, requiring both objective ancestry indicators and subjective identity measures. I leverage birthplace and ancestry data to construct objective Asian ancestry measures. I then analyze---among individuals that are objectively Asian---how bias and parental characteristics influence subjective racial identity reporting. I find that racial identity reporting negatively correlates with individual characteristics like parental education, and with environmental factors reflecting local discrimination levels. Specifically, I document that higher anti-Asian bias correlates with lower rates of Asian racial identification among Asian Americans.

Among individuals with Asian ancestry, I document that an increase in anti-Asian bias correlates with reduced Asian racial identity reporting. Specifically, a one standard deviation increase in bias corresponds to a statistically significant 9 percentage point decrease in Asian racial identification among first-generation immigrants, a statistically insignificant 5 percentage point decrease among second-generation individuals, and a statistically significant 8 percentage point decrease among third-generation Asian Americans. I conduct a heterogeneity analysis by family structure and show that: bias effects are strongest among children from mixed-race families, with a one standard deviation bias increase correlating with a 15 percentage point decrease in Asian identification among children of Asian fathers and White mothers, and a 10 percentage point decrease among children of White fathers and Asian mothers.

This paper contributes to multiple scholarly literatures. First, it extends the economics of identity framework by examining how racial stereotypes—both positive and negative—influence identity choices \autocite{akerlofEconomicsIdentity2000}. While previous work focused primarily on costs of minority identification, Asian Americans face a complex utility landscape where Asian identity can simultaneously signal competence (in educational contexts) and foreignness (in social settings).

The analysis connects to stratification economics research examining how racial landscapes shape economic outcomes. \textcite{darityEconomicsIdentityOrigin2006,darityPositionPossessionsStratification2022} establish how both historical and contemporary factors perpetuate racial stratification. This framework extends to Asian American experiences, where model minority stereotypes create unique forms of racialization distinct from other groups' experiences \autocite{goldsmithDarkLightSkin2007,hamiltonSheddingLightMarriage2009,dietteSkinShadeStratification2015}.

The paper also contributes to the literature documenting how political events and social attitudes influence racial identity formation. \textcite{masonStigmatizationRacialSelection2014} demonstrates how post-September 11 backlash affected Arab and Islamic American identity choices, while recent anti-Asian violence following COVID-19 creates similar pressures for Asian American communities. These external shocks to group status provide natural experiments for studying identity responsiveness to discrimination.

Within immigration and integration research, this work builds on studies examining how Asian Americans navigate assimilation processes \autocite{abramitzkyCulturalAssimilationAge2016}. Unlike European immigrant groups, Asian Americans face persistent ``perpetual foreigner'' stereotypes that complicate integration patterns \autocite{foukaImmigrantsAmericansRace2022}. The model minority stereotype creates additional complexity, as Asian identification may carry both benefits and costs depending on context \autocite{mengIntermarriageEconomicAssimilation2005}.

This paper most closely relates to research on racial identity fluidity and strategic ethnic identification \autocite{hadah2024hispanicidentity, antmanEthnicAttritionObserved2016,antmanIncentivesIdentifyRacial2015,antmanAmericanIndianCasinos2021}. However, while previous work focused primarily on Hispanic ethnic attrition, Asian American identity choices operate through different mechanisms due to distinct stereotypes, discrimination patterns, and socioeconomic profiles.\footnote{Racial identity fluidity occurs when individuals with Asian ancestry modify their racial self-identification in response to contextual factors.} Taking into consideration the identity flexibility that characterizes Asian American experiences, I investigate the determinants driving racial self-identification decisions. \textcite{hadah2024hispanicidentity} finds that bias and self-reported Hispanic identity are negatively associated among objectively Hispanic immigrants. I aim to examine how certain personal and environmental factors influence the complexity of endogenous racial identity among Asian Americans. The empirical analysis documents how observable characteristics—individual traits and societal attitudes—affect racial identity reporting among Asian Americans.

The rest of this paper proceeds as follows. First, I discuss the theoretical framework in section (\ref{sec:model}). Second, I describe the data sources in section (\ref{sec:data}). Third, I present the empirical approach and results in sections (\ref{sec:empstrat}) and (\ref{sec:results}). Fourth, I discuss robustness checks and alternative explanations in section (\ref{sec:robcheck}). Finally, I conclude in section (\ref{sec:conc}).

\section{Theoretical Framework}\label{sec:model}

I develop a theoretical framework for understanding racial identity choice that extends \textcite{akerlofEconomicsIdentity2000} to incorporate stereotype-specific costs and benefits. Unlike generic minority identification models, this framework recognizes that Asian Americans face unique utility trade-offs where racial identity can signal both positive attributes (academic achievement, work ethic) and negative characteristics (foreignness, social exclusion).

Formally, individual $i$ belongs to racial group $r_i \in \{A, W\}$, where $A$ represents Asian and $W$ represents White. Agent $i$'s utility depends on their actions and how those actions interact with their chosen racial identity $I_i$:

\begin{equation}
U_i = U_i(\pmb{a_i}, \pmb{a_{-i}}, I_i)\label{eq:util}
\end{equation}

Individual identity $I_i$ reflects personal actions, others' behaviors toward them, and societal expectations associated with their racial group:

\begin{equation}
I_i = I_i(\pmb{a_i}, \pmb{a_{-i}}; \pmb{S}_{r_{i}})\label{eq:identity}
\end{equation}

Where $\pmb{a_i}$ represents individual $i$'s actions, $\pmb{a_{-i}}$ captures others' actions affecting $i$'s identity (including anti-Asian bias), and $\pmb{S}_{r_{i}}$ denotes societal stereotypes and expectations associated with racial group membership.\footnote{This extends \textcite{akerlofEconomicsIdentity2000}'s proscription concept to encompass both negative stereotypes and positive model minority expectations.}

The key insight for Asian Americans is that $\pmb{S}_{A}$ includes both positive stereotypes (academic excellence, economic success) and negative ones (perpetual foreigner status, social exclusion). This creates context-dependent utility from Asian identification—beneficial in some settings (academic achievement contexts) but costly in others (social acceptance, political inclusion).

Individual $i$ selects actions $a_i$ to maximize utility given their racial group $r_i$, associated stereotypes $\pmb{S}_{r_{i}}$, and others' actions $\pmb{a_{-i}}$. The first-order condition becomes:

\begin{equation}
\frac{\partial U_i}{\partial a_i} + \frac{\partial U_i}{\partial I_i} \cdot \frac{d I_i}{d a_i} = 0\label{eq:foc}
\end{equation}

The solution $a_{i}^{\star}$ yields utility $U_{i}^{\star}$. Now suppose individuals can strategically choose their racial identity at cost $c$. They will switch identities when $\tilde{U_{i}}^{\star} \geq U_{i}^{\star} + c$, where $\tilde{U_{i}}^{\star}$ represents utility under the alternative racial identity.

Identity switching occurs when benefits $\tilde{U_{i}}^{\star} - U_{i}^{\star}$ exceed costs $c$. These net benefits are non-zero only when $\frac{d I_i}{d a_i} \neq 0$ and $\frac{\partial U_i}{\partial I_i} \neq 0$. This framework suggests empirical analysis should focus on: (1) individual characteristics affecting optimal actions under different racial identities, (2) contextual factors (anti-Asian bias) creating differential treatment by racial group, (3) populations with low switching costs $c$, and (4) groups whose utility significantly depends on racial identity.

From the empirical analysis, I investigate characteristics affecting individual actions under different identity choices from point (1). These characteristics include immigrant generation, mixed-race versus mono-racial family structure, etc. I also examine how anti-Asian bias influences identity choices. Finally, restricting analysis to individuals with low identity switching costs $c$ ensures the sample excludes populations unlikely to modify racial identification—for example, non-Asian Whites without Asian ancestry.

The model predicts that anti-Asian bias increases the utility differential between White and Asian identification, making identity switching more attractive. Mixed-race individuals face lower switching costs due to phenotypic ambiguity, while later-generation Asian Americans may find identity switching more feasible due to cultural assimilation.

\section{Data Sources and Measurement Strategy}\label{sec:data}

This section describes the datasets employed in the analysis. To examine relationships between social attitudes and Asian racial identity reporting, I require both subjective and objective Asian identity measures for selecting appropriate Asian American subgroups. I utilize the Integrated Public Use Microdata Series (IPUMS) Current Population Survey (CPS) \autocite{floodsarahIntegratedPublicUse2021a} with ancestry information to construct objective identity measures. I develop composite anti-Asian bias measures using \textcite{lubotskyInterpretationRegressionsMultiple2006}'s methodology to reduce attenuation bias.

\subsection{Measuring Asian Racial Identity}\label{subsec:cps}

I measure Asian racial identity using Current Population Survey (CPS) data from 2004-2021, enabling construction of objective Asian ancestry measures for minors living with parents. Following \textcite{antmanEthnicAttritionObserved2016,antmanEthnicAttritionAssimilation2020}, I utilize birthplace information for individuals, parents, and grandparents to create objective Asian ancestry indicators.\footnote{This approach parallels previous research but focuses on racial rather than ethnic categorization.} The methodology allows perfect identification of first-, second-, and third-generation Asian Americans (see Figure \ref{fig:diag} for visual representation). This approach enables construction of objective Asian ancestry measures for minors under age 17 living with parents.

The objective ancestry measure—distinct from subjective racial identification where respondents select ``Asian'' as their race—depends on birthplaces across three generations. The three identifiable generations include: 1) first-generation immigrants born in Asian countries with both parents also born in Asian countries, 2) second-generation individuals who are US-born citizens with at least one parent born in an Asian country, 3) third-generation Asian Americans who are US-born citizens with two US-born parents and at least one grandparent born in an Asian country.\footnote{I restrict first-generation cases to those whose parents were born in Asian countries to exclude US citizens born abroad to American parents.} The sample includes Asian Americans, first-, second-, and third-generation immigrants aged 17 and younger living with parents between 2004 and 2021. Summary statistics appear in Table (\ref{tab:sumstat1}).

While CPS relies on household respondents (parents or caregivers) to report children's racial identity, this proxy reporting likely reflects children's actual identity since parents significantly influence identity formation. \textcite{duncanIntermarriageIntergenerationalTransmission2011} support this perspective, demonstrating no variation in Asian identification based on household respondent type. The data confirms consistent Asian identity reporting regardless of whether mother (72\%), father (72\%), or child/other caregiver (87\%) serves as respondent, as shown in Table \ref{tab:hispbyproxy}.\footnote{According to CPS guidelines, household respondents must be at least 15 years old with sufficient household knowledge. When the proxy is `self,' the respondent ranges from 15 to 17 years old.} Since my analysis compares high and low bias states, estimates remain valid provided reporting patterns don't systematically differ between these contexts.

The overall sample comprises 49\% females, with 65\% self-reporting Asian racial identity—answering affirmatively to ``what is your race.'' Average age equals 8.4 years. Approximately 52\% of mothers and 52\% of fathers hold college degrees. Additional summary statistics for the overall sample and each generation appear in Table (\ref{tab:sumstat1}).

Using parental and grandparental birthplaces, I objectively identify ethnic ancestry and categorize different family types. For second-generation children, parental birthplaces create three objective categories:
\begin{enumerate}
\item Objectively Asian-father-Asian-mother (AA)
\item Objectively Asian-father-White-mother (AW)  
\item Objectively White-father-Asian-mother (WA)
\end{enumerate}

Similarly, grandparental birthplaces create 15 objective categories for third-generation children: (1) objectively Asian paternal grandfather-Asian paternal grandmother-Asian maternal grandfather-Asian maternal grandmother (AAAA); (2) objectively White paternal grandfather-Asian paternal grandmother-Asian maternal grandfather-Asian maternal grandmother (WAAA); (3) objectively Asian paternal grandfather-White paternal grandmother-Asian maternal grandfather-Asian maternal grandmother (AWAA), etc.

My analysis employs a US population subsample; Table (\ref{tab:hispbygen}) demonstrates sufficient observations across generations. Consistent with literature on racial identity fluidity among Asian Americans, I document significant attrition among third-generation Asian Americans.\footnote{\textcite{duncanIdentifyingLaterGenerationDescendants2018,duncanSocioeconomicIntegrationImmigrant2018, antmanEthnicAttritionObserved2016,antmanEthnicAttritionAssimilation2020} document substantial identity attrition among various groups.} Table (\ref{tab:hispbygen}) displays these patterns: most first- and second-generation Asian Americans racially self-identify as Asian. Among first-generation Asian Americans, 96\% self-report Asian racial identity. Among second-generation Asian Americans, 73\% self-identify as Asian, while 31\% of third-generation Asian Americans choose Asian racial identification. Attrition among second- and third-generation Asian Americans primarily occurs among children from interracial families.

\subsection{Measuring Anti-Asian Sentiment}

I construct anti-Asian sentiment measures using implicit association tests, American National Election Studies, and hate crimes targeting Asian Americans from 2004-2021. The implicit association test measures conceptual associations—for example, linking Asian Americans with negative stereotypes—and evaluative responses. Respondents rapidly categorize words into screen-displayed categories. Figure (\ref{fig:iatexamples}) shows examples from Harvard's Project Implicit skin tone test.

I employ Asian-focused implicit association test data to construct anti-Asian prejudice proxies \autocite{greenwaldMeasuringIndividualDifferences1998}. This measure has extensive social science applications, particularly in psychology. Previous research demonstrates IAT score manipulation difficulty \autocite{egloffPredictiveValidityImplicit2002}.

The IAT measures bias direction and magnitude while capturing unconscious biases individuals may be unwilling to report. Meta-analysis of over 122 IAT studies by \textcite{greenwaldMeasuringIndividualDifferences1998} finds significantly higher predictive validity for IAT compared to self-report measures. However, some research questions IAT predictive validity claims.\footnote{Research correlates IAT tests with economic outcomes \autocite{chettyRaceEconomicOpportunity2020,gloverDiscriminationSelfFulfillingProphecy2017}, voting behavior \autocite{friesePredictingVotingBehavior2007}, and health \autocite{leitnerRacialBiasAssociated2016}. IAT participation is voluntary, potentially creating selection bias. However, IAT-reflected bias serves as a prejudiced attitude proxy \autocite{chettyRaceEconomicOpportunity2020}.} Implicit Association Tests may not reliably measure or predict implicit prejudice or biased behaviors. Research shows implicit biases experience minor, temporary intervention-induced changes. Additionally, implicit bias fails to predict dictator game contributions or social pressure susceptibility, highlighting distinctions between implicit bias and biased actions \autocite{arkesAttributionsImplicitPrejudice2004,forscherMetaanalysisProceduresChange2019,leeDoesImplicitBias2018}. Therefore, I supplement IAT with explicit bias measures from American National Election Studies (ANES) to construct composite bias measures.

I develop another racial animus proxy using ANES surveys \autocite{anes2021} measuring discrimination against Black Americans. ANES, conducted since 1948, enjoys widespread political science usage. The survey examines attitudes toward different racial groups, voting intentions, and political questions. I employ several 2004-2020 ANES questions measuring racial animus. The racial animus index averages responses across multiple animus-measuring questions.\footnote{Questions parallel those used by \textcite{charlesPrejudiceWagesEmpirical2008}: (1) ``Conditions Make it Difficult for Blacks to Succeed'', (2) ``Blacks Should Not Have Special Favors to Succeed'', (3) ``Blacks Must Try Harder to Succeed'', (4) ``Blacks Gotten Less than They Deserve Over the Past Few Years'', and (5) ``Feeling Thermometer Toward Asians.''}

Finally, I incorporate Uniform Crime Reports (UCR) data quantifying hate crimes against Asian Americans \autocite{ucrbook}. Hate crime data provides tangible racially-motivated aggression and discrimination measures. Combined with implicit and explicit bias measures, this enables comprehensive prejudice understanding across states. This multidimensional approach—implicit bias, explicit bias, and hate crime statistics—offers fuller racial prejudice landscape understanding.

To reduce attenuation bias and measurement error, I follow \textcite{lubotskyInterpretationRegressionsMultiple2006} constructing composite bias measures using IAT, ANES racial animus measures, and hate crimes against Asian Americans.\footnote{Additional methodological details appear in the Data Online Appendix, Section~\ref{sub:lw-bias}.} Figure (\ref{fig:skiniat}) graphically represents bias measures over time in most and least biased locations. Figure (\ref{fig:Asian-twostates}) shows Asian racial identity reporting in the two most and least biased locations. Lower scores indicate less bias; higher scores indicate greater racial animus. One standard deviation bias increases equivalent to moving from Washington, DC, or Vermont to North Dakota in 2020. State-level average bias over time appears in Figure (\ref{fig:skiniat-maps}) maps, with overall 2004-2021 averages in Figure (\ref{fig:iat-map-all}).

\section{Empirical Approach and Findings}\label{sec:empstrat}

To understand associations between Asian racial self-identification and anti-Asian bias, I estimate regressions of the following form for each generation $g$:

\begin{align}
A_{ist}^g &= \beta_1^g AntiAsianBias_{st} + \beta_2^g DadCollegeGrad_{ist} + \beta_3^g MomCollegeGrad_{ist} \nonumber \\ 
            &+ \beta_4^g Women_{ist} + X_{ist}^g\pi + \gamma_{rt} 
           + \varepsilon_{ist}; 
           \text{where } g \in \{1,2,3\} \label{eq:identity_reg_bias}
\end{align}

Where $A_{ist}^g$ represents self-reported Asian racial identity of person $i$ in state $s$ at interview time $t$, $AntiAsianBias_{st}$ represents average anti-Asian bias in state $s$ at time $t$, $DadCollegeGrad_{ist}$ and $MomCollegeGrad_{ist}$ are indicator variables equaling one if father or mother graduated college, $Women_{ist}$ indicates sex, and $X_{ist}$ represents a control vector.\footnote{Controls include quartic age, Asian population fraction in state $s$, parent types (WA, AW, or AA), grandparent types (AAAA, AAAW, etc.), and generation dummy variables.} Additionally, $\gamma_{rt}$ represents region-time fixed effects controlling for region $\times$ year specific shocks.\footnote{I exclude state fixed effects due to insufficient within-state bias variation.} Region $\times$ year controls also account for systematic regional differences in overall Asian American populations and anti-Asian bias, even with temporal variation. Throughout the analysis, I cluster standard errors at state level accounting for error term $\varepsilon_{ist}$ correlation within states over time.

Since specifications include region $\times$ year $\gamma_{rt}$, the $\beta_1^g$ coefficient summarizes individual $i$ responsiveness to anti-Asian bias changes in their residence state. In other words, $\beta_1^g$ captures associations between Asian racial identity reporting and anti-Asian bias across states within Census division regions. Additionally, $\gamma_{rt}$ fixed effects account for regional and national bias trends over time. Consequently, $\beta_1^g$ provides correlations between Asian racial identity reporting and anti-Asian bias beyond national and regional bias trends. If individuals in regional states responded similarly to bias changes, then $\beta_1^g$ equals zero.

\section{Results}\label{sec:results}

The empirical analysis provides consistent evidence that anti-Asian bias negatively correlates with Asian racial identity reporting. These relationships are strongest among individuals with greatest identity flexibility—mixed-race individuals and later-generation Asian Americans.

I report main results from estimating equation (\ref{eq:identity_reg_bias}) in Figure (\ref{plot01-regression-gen}). I present results estimating the main specification for all generations in panel (A) and for first-, second-, and third-generation subsamples in panels (B), (C), and (D), respectively. Anti-Asian bias and Asian racial identity reporting exhibit negative associations. One standard deviation anti-Asian bias increases correlate with 9 percentage point decreases in Asian racial identity reporting. Among first- and second-generation Asian Americans, one standard deviation anti-Asian bias increases associate with 5 and 8 percentage point decreases in Asian racial identity reporting. The first-generation coefficient lacks statistical significance, but confidence intervals remain predominantly negative. Among third-generation Asian Americans, one standard deviation anti-Asian bias increases associate with 8 percentage point decreases in Asian racial identity reporting. Moreover, I find that---among all objectively Asian individuals---having a college-educated father or mother increases Asian racial identity reporting by 1 percentage point.

I report identical regression results for second-generation immigrant subsamples by parent type—interracial versus endogamous parents—in Figure (\ref{plot01-regression-byparent}). I present main specification results for second-generation immigrants in panel (A) and for AA, AW, and WA children subsamples in panels (B), (C), and (D), respectively. Children from interracial families show greater bias influence. One standard deviation anti-Asian bias increases associate with 5 percentage point decreases in Asian racial identity reporting among endogamous parent children—estimates are statistically insignificant. However, one standard deviation anti-Asian bias increases associate with 15 percentage point decreases in Asian racial identity reporting among Asian father-White mother children, and 10 percentage point decreases among White father-Asian mother children.

I also report regression results for third-generation immigrant subsamples by Asian grandparent numbers in Table (\ref{regtab-bygrandparents}). Overall anti-Asian bias effects on different Asian American children types are negative but mostly statistically insignificant. One standard deviation anti-Asian bias increases associate with 69 percentage point decreases in Asian racial identity reporting among Asian American children with three Asian-born grandparents.

\section{Robustness Checks and Alternative Explanations}\label{sec:robcheck}

This section explores empirical relationships between anti-Asian bias and interracial marriages, plus migration patterns among second-generation Asian Americans as robustness checks for main analysis and proxy response effects. I examine anti-Asian bias impacts on interracial marriage likelihood, focusing on interracial couples, and Asian American migration decisions within the United States.

I investigate relationships between anti-Asian bias and interracial marriages using the following regression specification:

\begin{align}
interracial_{ist}^2 &= \beta_1^2 AntiAsianBias_{st} + X_{ist}^2\pi + \gamma_{rt} 
            + \varepsilon_{ist}  \label{eq:inter-interracial} 
\end{align}

Where $interracial_{ist}^2$ indicates interracial couples, i.e., Asian husband-White wife or White husband-Asian wife. $AntiAsianBias_{st}$ represents average anti-Asian bias in state $s$ at time $t$, and $X_{ist}^2$ represents partner-specific controls affecting marriage matching including wife's and husband's education, age, and years since US immigration.

I present equation (\ref{eq:inter-interracial}) estimation results in Table (\ref{regtab-logit-02}). One standard deviation anti-Asian bias increases raise interracial parent probabilities by 4 percentage points. Breaking down analysis by couple ethnicity: one standard deviation anti-Asian bias increases associate with 1 percentage point decreases in Asian husband-White wife marriage chances. One standard deviation anti-Asian bias increases associate with 3 percentage point increases in Asian wife-White husband chances. Anti-Asian bias and interracial marriage positive correlations may result from Asian Americans in high-bias states aiming to reduce children's Asian ethnicity signal likelihood. For example, Asian American women in high-bias states might marry non-Asian White husbands, providing children non-Asian surnames.

I also investigate relationships between anti-Asian bias and migration. Since CPS doesn't report birth states, I use 2004-2021 Censuses constructing second-generation Asian American samples \autocite{floodsarahIntegratedPublicUse2021}. I construct mover variables indicating whether second-generation Asian Americans moved from birth states to other states. I use the following models estimating relationships between anti-Asian bias and migration:

\begin{align}
BirthPlaceMigration_{ist}^2 &= \beta_1^2 AntiAsianBias_{st} 
                   + X_{ist}^2\pi + \gamma_{rt} 
                   + \varepsilon_{ist} \label{eq:migration-3} \\
BirthPlaceMigration_{ilb}^2 &= \beta_1^2 AntiAsianBias_{lb} 
                   + X_{ilb}^2\pi + \gamma_{lb} 
                   + \varepsilon_{ilb} \label{eq:migration-4}
\end{align}

Where $BirthPlaceMigration_{ist}^2$ indicates whether person $i$ in state $s$ at interview $t$ lives in states different from birth states. $BirthPlaceMigration_{ilb}^2$ indicates whether person $i$ in birthplace $l$ doesn't currently live in the same state as birth year $b$. Analysis, restricted to second-generation Asian Americans with both Asian-born parents, uses equations (\ref{eq:migration-3}) and (\ref{eq:migration-4}).

I employ two approaches defining bias variables studying relationships between bias and migration variables. First specification from equation (\ref{eq:migration-3}) estimates relationships between average bias at interview time $t$ in state $s$ and $BirthPlaceMigration_{ist}^2$. Second specification from equation (\ref{eq:migration-4}) estimates relationships between average bias in birth state $l$ at birth year $b$ and $BirthPlaceMigration_{ilb}^2$.

I also estimate whether Asian-identifying individuals tend moving from high-bias to low-bias states using:

\begin{align}
Y_{ist} &= \beta_0 + \beta_1^2 Asian_{ist} +
                   X_{ist}^2\pi
                   + \varepsilon_{ist} \label{eq:migration-5}
\end{align}

Where $Y_{ist} \equiv AntiAsianBias_{ist} - AntiAsianBias_{ilb}$, $AntiAsianBias_{ist}$ represents $i$'s anti-Asian bias in state $s$ at interview time $t$, and $AntiAsianBias_{ilb}$ represents $i$'s anti-Asian bias in birth state $l$ at birth year $b$. Analysis restricts to second-generation Asian Americans with both Asian-born parents who migrated from birth state $b$ to another state $s$.

I show the results of estimating Equations (\ref{eq:migration-3}), (\ref{eq:migration-4}), and (\ref{eq:migration-5}) in Table (\ref{regtab-mig-01}) columns (1), (2), and (3) respectively. Among second-generation immigrants, no significant correlations exist between anti-Asian bias and migration decisions. Among second-generation Asian American movers, those self-reporting Asian racial identity live in states with 0.06 standard deviations greater bias than birth states. While this result shows selection into more biased states among second-generation immigrants, it doesn't affect main results showing correlations between anti-Asian bias and Asian racial identity reporting. Since Asian-identifying individuals are movers, my bias and Asian racial identity reporting relationship assessments might underestimate bias effects.

These findings indicate negative correlations between anti-Asian bias and Asian racial identity reporting among Asian Americans. While I don't aim establishing causal anti-Asian bias effects on Asian racial identity reporting, I illustrate correlations between bias and racial identification. These correlations suggest that depending on state bias levels, racial gaps relying on self-reported identity might overestimate or underestimate discrimination effects.

Several analysis concerns exist. First, Current Population Survey (CPS) self-reported identity comes from household respondents—parents or adult caregivers. Thus, 'self-reported' racial identity might not reflect children's true identity. I view parent- or caregiver-reported identity as accurate child identity representation since parents essentially shape children's self-concepts. Also, I compare high- and low-bias states for analysis. Estimates remain unchanged if self-reporting likelihood doesn't differ between these states.

Moreover, \textcite{duncanIntermarriageIntergenerationalTransmission2011} show reported Hispanic identification doesn't vary with household respondent identity. Additionally, I present main Asian racial identity reporting effects by household respondent in Table (\ref{tab:hispbyproxy}). Main reported Asian racial identity effects equal 72 percentage points when mothers serve as proxies, 72 percentage points when fathers serve as proxies, and 87 percentage points when children or other caregivers serve as household respondents.\footnote{According to Current Population Survey (CPS) guidelines, household respondents must be at least 15 years old with sufficient household knowledge. When the proxy is `self,' respondents range from 15 to 17 years old.}

A second concern involves IAT voluntary participation and non-representative population sampling. While I don't claim IAT anti-Asian bias proxies represent populations, \textcite{egloffPredictiveValidityImplicit2002} demonstrate manipulation difficulty. Several studies correlate IAT with economic outcomes \autocite{chettyRaceEconomicOpportunity2020,gloverDiscriminationSelfFulfillingProphecy2017}, voting behavior \autocite{friesePredictingVotingBehavior2007}, decision-making \autocite{bertrandImplicitDiscrimination2005,carlanaImplicitStereotypesEvidence2019}, and health \autocite{leitnerRacialBiasAssociated2016}. Another concern involves IAT test-taker characteristic changes over time, creating non-identical samples. I address this concern including non-parametric region $\times$ year fixed effects controlling systematic test-taker characteristic differences between regions. These changes remain controlled provided test-taker characteristic differences don't vary across states within regions. Most importantly, I use ANES racial animus measures and hate crimes against Asian Americans constructing composite bias measures reducing measurement error using \textcite{lubotskyInterpretationRegressionsMultiple2006}.

Another concern involves reverse causality between greater Asian American or Black populations in states and bias levels. Greater Asian American populations in states might affect resident bias levels. For example, more Asian Americans in California or Black Americans in Louisiana could affect California and Louisiana resident bias. To demonstrate this isn't occurring, I provide Figure (\ref{scatter-plot-1}) evidence. Figure (\ref{scatter-plot-1}) plots self-reported Asian American state percentages at specific years against average anti-Asian bias in identical states during those years. I find no correlations between anti-Asian bias and Asian American state populations, making reverse causality unlikely.

Finally, bias (prejudice) and Asian racial identity reporting relationship estimators could be biased if non-Asian-identifying individuals migrate to more prejudiced states. I've shown above this isn't occurring (Table \ref{regtab-mig-01}). I find no evidence of relationships between migration decisions and anti-Asian bias. Additionally, I find those reporting Asian racial identity moved from less biased birthplaces and lived in more biased states at survey times. Thus, my results might underestimate relationships between anti-Asian bias and Asian racial identity reporting.

\section{Conclusion}\label{sec:conc}

As American society becomes increasingly multiracial, racial identity choices will significantly influence political representation, resource allocation, and social cohesion. Understanding identity determinants are particularly important for researchers studying discrimination's role in racial economic gaps. This paper demonstrates how individual characteristics and anti-Asian sentiment influence racial identity reporting among Asian Americans.

I find that individuals with Asian ancestry are significantly less likely to racially identify as Asian in states with increases anti-Asian bias. The relationships between Asian racial identity reporting and anti-Asian bias among first-generation immigrants show one standard deviation bias increases correlating with 9 percentage point decreases in Asian racial identity reporting; results are statistical significance. Relationships between Asian racial identity reporting and anti-Asian bias are more prominent among second-generation immigrants, where one standard deviation bias increases correlate with 5 percentage point decreases in Asian racial identity reporting. Among third-generation Asian Americans, one standard deviation anti-Asian bias increases correlate with 8 percentage point decreases in Asian racial identity reporting.

Additionally, anti-Asian bias produces more substantial effects among second-generation immigrant children from mixed-race families. One standard deviation anti-Asian bias increases correlate with 15 percentage point decreases in Asian racial identity reporting among second-generation Asian American children of Asian fathers and White mothers, and 10 percentage point decreases among children of White fathers and Asian mothers. I also find anti-Asian bias positively correlates with interracial marriage and shows no correlation with migration decisions.

These results are important due to consequences for correct Asian American and minority enumeration, integration patterns, and social mobility. They indicate anti-Asian bias could significantly affect how economists estimate racial gaps. Most race and ethnicity research relies on self-reported racial and ethnic identity measures. Since anti-Asian bias negatively correlates with Asian racial identity reporting, characteristics of those avoiding Asian racial identification could produce important consequences. For example, if individuals whose identities are most likely affected by bias represent the most educated, then racial gaps will be overestimated in the most biased states. Furthermore, identity decisions likely profoundly affect people's choices, investments, and well-being.

Moreover, this study could encourage further research into relationships between bias and self-reported identities for other groups. Analysis of bias effects on self-reported identity could apply to other groups. For example, researchers could estimate bias effects on sexual minority identities and other ethnic and racial minorities such as Black, Native American, and Arab American populations in the United States. Researchers could also explore outcome differences between ethnic and racial minorities who self-report versus those who don't using restricted administrative data.

The research opens several avenues for future investigation. First, scholars could examine how recent anti-Asian violence following COVID-19 has influenced identity patterns, providing natural experiments in bias effects. Second, researchers might explore identity choices in specific institutional contexts like college admissions or workplace advancement, where model minority stereotypes create complex incentive structures. Third, analysis could extend to other Asian American subgroups, recognizing that Chinese, Korean, Vietnamese, and other communities face distinct stereotypes and discrimination patterns.

Understanding strategic racial identification among Asian Americans is essential for designing effective anti-discrimination policies and accurately measuring racial equity progress. As debates over affirmative action, immigration, and racial justice continue evolving, recognizing how Asian Americans navigate identity choices becomes increasingly critical for promoting inclusive and equitable outcomes.