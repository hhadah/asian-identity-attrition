\begin{table}[H]
\centering\centering
\caption{Relationship Between Bias and Migration \label{regtab-mig-01}}
\centering
<<<<<<< HEAD
\resizebox{0.7\textwidth}{!}{
=======
\resizebox{0.75\textwidth}{!}{
>>>>>>> 316a54059d1a2890f868822370e4e12cafadaa07
\begin{threeparttable}
\begin{tabular}[t]{lccc}
\toprule
  & \specialcell{(1) \\ Migrated from \\ Birth Place} & \specialcell{(2) \\ Migrated from \\ Birth Place} & \specialcell{(3) \\ $Bias_{ist} - Bias_{ilb}$}\\
\midrule
$Bias_{st}$ & 0.13* &  & \\
 & (0.07) &  & \\
$Bias_{lb}$ &  & -0.03 & \\
 &  & (0.17) & \\
Asian &  &  & 0.02\\
 &  &  & (0.04)\\
Female & 0.00 & -0.01 & 0.00\\
 & (0.00) & (0.00) & (0.02)\\
College Graduate: Mother & 0.01*** & 0.00 & -0.01\\
 & (0.00) & (0.01) & (0.03)\\
College Graduate: Father & -0.03*** & -0.03*** & 0.03\\
 & (0.01) & (0.01) & (0.02)\\
\midrule
Observations & 73,563 & 41,641 & 2,075\\
Mean & 0.15 & 0.15 & -0.1\\
Year $\times$ Region FE & X &  & \\
Birthyear $\times$ Birth Region FE &  & X & \\
\bottomrule
\multicolumn{4}{l}{\rule{0pt}{1em}* p $<$ 0.1, ** p $<$ 0.05, *** p $<$ 0.01}\\
\end{tabular}
\begin{tablenotes}
\small
\item[1] \footnotesize{Each column is an estimation of equations (\ref{eq:migration-3}) in column (1), 
                      (\ref{eq:migration-4}) in column (2), and
                      (\ref{eq:migration-5}) in column (3).}
\item[2] \footnotesize{Column (1) is a regression where the left hand side variable is 
                      a dummy variable that is equal to one if a person migrated from the state
                      were born in and the right hand side variable is bias the year of survey.
                      Column (2) is a regression where the left hand side variable is 
                      a dummy variable that is equal to one if a person migrated from the state
                      were born in and the right hand side variable is bias the year of birth in the state of birth.
                      Column (3) is a regression where the left hand side variable is 
                      the difference between state-level bias during the year of the survey in the current state the 
                      respondent is living in, and state-level bias during the year of birth in the state of birth 
                      and the right hand side variable is self-reported Asian identity. This regression captures
                      the selection of those that self-reported Asian identity into states with different levels of bias.
                      I include controls for sex, quartic age, parental education, fraction of Asians in a state, and region × year fixed effects.
                      Standard errors are clustered on the state level.}
\item[3] \footnotesize{The samples include children ages 17 and below who live in intact families. 
                      Native-born second-generation Asian immigrant children with both
                      parents born in a Asian country. The sample in the column (3) regression is further restricted to only those that migrated from their birth state.}
\item[4] \footnotesize{Data source is the 2004-2021 Census Data.}
\end{tablenotes}
\end{threeparttable}}
\end{table}
