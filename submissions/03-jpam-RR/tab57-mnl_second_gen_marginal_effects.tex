\begin{table}[H]
\centering
\caption{Multinomial Logit — Second Generation (AMEs)}
\centering
\resizebox{\ifdim\width>\linewidth\linewidth\else\width\fi}{!}{
\begin{threeparttable}
\begin{tabular}[t]{lrrrrrr}
\toprule
variable\_label & est\_ci\_Asian\_only & est\_ci\_White\_only & est\_ci\_Asian\_and\_White & p.value\_Asian\_only & p.value\_White\_only & p.value\_Asian\_and\_White\\
\midrule
Anti-Asian Bias & -0.0668 [-0.0668, -0.0668] & 0.0879 [0.0879, 0.0879] & -0.0211 [-0.0211, -0.0211] & 0 & 0 & 0\\
College Graduate: Father & 0.0017 [0.0017, 0.0017] & -0.0113 [-0.0113, -0.0113] & 0.0097 [0.0097, 0.0097] & 0 & 0 & 0\\
College Graduate: Mother & 0.0255 [0.0255, 0.0255] & -0.0238 [-0.0238, -0.0238] & -0.0017 [-0.0017, -0.0017] & 0 & 0 & 0\\
Female & 0.0019 [0.0019, 0.0019] & 0.0118 [0.0118, 0.0118] & -0.0137 [-0.0137, -0.0137] & 0 & 0 & 0\\
\bottomrule
\end{tabular}
\begin{tablenotes}
\item[1] footnotesize{Average Marginal Effects (AME) with 95% CI.}
\item[2] footnotesize{Standard errors clustered at the state level.}
\item[3] footnotesize{Reference category is embedded in the multinomial baseline (Asian only).}
\item[4] footnotesize{Models include region-year fixed effects.}
\end{tablenotes}
\end{threeparttable}}
\end{table}
