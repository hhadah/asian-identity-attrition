\begin{table}[H]
\centering
\caption{CPS Summary Statistics \label{tab:sumstat1}}
\centering
\begin{threeparttable}
\begin{tabular}[t]{lcccc}
\toprule
\multicolumn{1}{c}{ } & \multicolumn{1}{c}{\textbf{Overall}} & \multicolumn{3}{c}{\textbf{By Generation}} \\
\cmidrule(l{3pt}r{3pt}){2-2} \cmidrule(l{3pt}r{3pt}){3-5}
\textbf{Characteristic} & \makecell[c]{\textbf{All Sample} \\N = 318,404} & \makecell[c]{\textbf{First} \\N=40,033} & \makecell[c]{\textbf{Second} \\N=199,294} & \makecell[c]{\textbf{Third} \\N=79,077}\\
\midrule
Female & 0.49 & 0.53 & 0.49 & 0.49\\
Asian & 0.65 & 0.96 & 0.73 & 0.31\\
Age & 8.4 (5.1) & 10.9 (4.5) & 8.3 (5.1) & 7.7 (5.0)\\
College Graduate:\ \ 	 Father & 0.52 & 0.59 & 0.52 & 0.50\\
College Graduate:\ \ 	 Mother & 0.52 & 0.56 & 0.51 & 0.52\\
Total Family Income\ \ 	 (1999 dollars) & 87,031 (84,797) & 75,815 (74,489) & 88,295 (88,411) & 89,436 (80,051)\\
\bottomrule
\end{tabular}
\begin{tablenotes}
\item[1] The samples include children ages 17 and below who live in intact families. First-generation Asian immigrant children that were born in a Asian country. Native-born second-generation Asian immigrant children with at least one parent born in a Asian country. Finally, native-born third generation Asian immigrant children with native-born parents and at least one grand parent born in a Asian country.
\item[2] Data source is the 2004-2021 Current Population Survey.
\end{tablenotes}
\end{threeparttable}
\end{table}
