%%%%%%%%%%%%%%%%%%%%%%%%%%%%%%%%%%%
% Main Text
%%%%%%%%%%%%%%%%%%%%%%%%%%%%%%%%%%%

\section{Introduction}\label{sec:intro}

Asian Americans represent the fastest-growing racial group in the United States, yet their experiences with discrimination and identity formation remain underexplored in economic research.\footnote{The 2020 Census counted more than 20 million Asian Americans---6.4 percent of the population---nearly double the number counted two decades earlier \autocite{floodsarahIntegratedPublicUse2021a}. The Asian American population numbers are based on the author's calculations from the Current Population Survey and US Census data.} Unlike other minority groups, Asian Americans occupy a distinctive position in America's racial hierarchy—simultaneously experiencing discrimination while being characterized through the ``model minority'' stereotype. This dual status creates complex incentives around racial identity choices that fundamentally differ from other groups' experiences, as Asian identification can signal both academic excellence and perpetual foreignness.

An extensive literature has documented Asian-White gaps in various outcomes \autocite{chiswick1983analysis, duleep2012economic, hilger2016upward, arabsheibani2010asian}, yet the role of identity selection in shaping these disparities remains understudied. The challenge lies in defining and measuring racial identity, particularly when individuals possess agency in how they racially self-identify. To the extent that reporting Asian racial identity represents a strategic choice influenced by local discrimination, measured gaps may systematically vary across geographic contexts in ways that previous research has not fully explored.

Various contextual factors, including anti-Asian sentiment and stereotype threat, can influence how individuals navigate their racial identity choices. Recent events have brought renewed attention to how external hostility shapes Asian American experiences. In this paper, I examine the determinants of Asian racial identity reporting and analyze how Asian Americans strategically select between Asian and White racial identities. Specifically, I investigate how anti-Asian bias shapes decisions to racially identify, or not, as Asian American.

This paper has important implications for several reasons. First, if individuals respond to prejudice by avoiding Asian racial identification, conventional analyses of racial gaps may systematically overestimate disparities in the most prejudiced areas. This would lead to misunderstanding of both the extent and geographic distribution of discrimination against Asian Americans. Second, identity choices may influence measured labor market trajectories among racial groups, potentially making Asian American integration appear more successful than reality suggests, thereby reinforcing model minority stereotypes that obscure genuine barriers faced by Asian American communities. Third, strategic identity reporting affects the enumeration of Asian American populations, with implications for political representation, resource allocation, and the design of policies aimed at addressing racial inequities.

I explore how individual characteristics and societal attitudes toward Asian Americans influence racial identity reporting. I utilize identity and ancestry data from the Current Population Survey (CPS) combined with measures of anti-Asian bias derived from Harvard's Project Implicit Association Test (IAT), the American National Election Studies (ANES), and hate crimes targeting Asian Americans.\footnote{The IAT data comes from Harvard's Project Implicit \autocite{greenwaldMeasuringIndividualDifferences1998}. Implicit bias measures have gained prominence in economics, with IAT scores correlating with economic outcomes \autocite{chettyRaceEconomicOpportunity2020,gloverDiscriminationSelfFulfillingProphecy2017}, voting patterns \autocite{friesePredictingVotingBehavior2007}, and health disparities \autocite{leitnerRacialBiasAssociated2016}.} I ground my analysis in a theoretical framework of \textcite{akerlofEconomicsIdentity2000}, explicitly modeling how external prejudice creates differential utility from identity choices and establishing conditions under which individuals strategically modify their racial self-presentation.

Measuring identity choices outside of lab settings presents is challenging, requiring both objective ancestry indicators and subjective identity measures. I leverage birthplace and ancestry data to construct objective Asian ancestry measures, then analyze deviations between objective ancestry and subjective racial identification. The analysis reveals that racial identity reporting negatively correlates with individual characteristics like parental education, and with environmental factors reflecting local discrimination levels.

Among individuals with Asian ancestry, I document that heightened anti-Asian bias correlates with reduced Asian racial identity reporting. Specifically, a one standard deviation increase in bias corresponds to a statistically significant 9 percentage point decrease in Asian racial identification among first-generation immigrants, a statistically insignificant 5 percentage point decrease among second-generation individuals, and a statistically significant 8 percentage point decrease among third-generation Asian Americans. The analysis by family structure reveals additional heterogeneity: bias effects prove strongest among children from mixed-race families, with a one standard deviation bias increase correlating with a 15 percentage point decrease in Asian identification among children of Asian fathers and White mothers, and a 10 percentage point decrease among children of White fathers and Asian mothers. These patterns suggest that economically successful Asian Americans—those with higher education and wealth—may strategically avoid Asian racial identification, leading research using subjective measures to underestimate Asian-White gaps in highly prejudiced areas.

This research contributes to multiple scholarly literatures in economics. First, it extends the economics of identity framework by examining how racial stereotypes—both positive and negative—influence identity choices \autocite{akerlofEconomicsIdentity2000}. Building on \textcite{charnessSocialIdentityGroup2020} and \textcite{atkinHowWeChoose2021}, I show that Asian Americans face a complex utility landscape where Asian identity can simultaneously signal competence (in educational contexts) and foreignness (in social settings).

The analysis connects to stratification economics research examining how racial hierarchies shape economic outcomes \autocite{darityEconomicsIdentityOrigin2006,darityPositionPossessionsStratification2022}. This framework extends to Asian American experiences, where model minority stereotypes create unique forms of racialization distinct from other groups' experiences \autocite{goldsmithDarkLightSkin2007,hamiltonSheddingLightMarriage2009,dietteSkinShadeStratification2015}.

While my formal theoretical model provides a logical framework for understanding how bias affects racial identity, the behavioral literature on racial identity and assimilation offers crucial empirical insights that complement this theoretical approach. Research in this tradition emphasizes how racial identity is not merely a cognitive construct but is actively performed, negotiated, and reconstructed through daily interactions and life experiences \autocite{waters1990ethnic}. \textcite{telles2008generations} study of Mexican Americans demonstrates how individuals strategically adapting their racial presentations across different social contexts while maintaining core identity elements across generations. Similarly, behavioral studies have documented how discrimination experiences shape identity salience and group attachment, with individuals developing adaptive strategies that range from ethnic distancing to reactive ethnicity depending on situational factors \autocite{zhou1997segmented}. This behavioral perspective reveals that racial identity operates as both a response to external categorization and an active process of boundary maintenance \autocite{cornell2006ethnicity}. While this literature has primarily relied on qualitative observations and ethnographic methods to document identity flexibility, the present analysis advances this understanding by quantifying these strategic choices through systematic comparison of objective ancestry measures with subjective racial identification across varying environmental contexts.

The paper also contributes to research on discrimination in economic contexts. \textcite{bertrandAreEmilyGreg2004} and \textcite{charlesPrejudiceWagesEmpirical2008} demonstrate how prejudice affects labor market outcomes, while recent work by \textcite{bursztynImmigrantNextDoor2022} explores how long-term exposure shapes attitudes. My analysis extends this literature by examining how discrimination influences the fundamental question of racial self-identification.

Within immigration and integration research, this work builds on studies examining how Asian Americans navigate assimilation processes \autocite{abramitzkyCulturalAssimilationAge2016,abramitzkyNationImmigrantsAssimilation2014}. Unlike European immigrant groups, Asian Americans face persistent ``perpetual foreigner'' stereotypes that complicate integration patterns regardless of generational status \autocite{foukaImmigrantsAmericansRace2022}. The model minority myth creates additional complexity, as Asian identification may carry both benefits and costs depending on context \autocite{mengIntermarriageEconomicAssimilation2005}.

This paper most closely relates to recent economic research on racial identity fluidity and strategic ethnic identification \autocite{hadah2024hispanicidentity, antmanEthnicAttritionObserved2016,antmanIncentivesIdentifyRacial2015,antmanAmericanIndianCasinos2021}. However, while previous work focused primarily on Hispanic ethnic attrition, Asian American identity choices operate through different economic mechanisms due to distinct stereotypes, discrimination patterns, and socioeconomic profiles. The concept of ``racial identity flexibility'' among Asian Americans reflects both the economic advantages and constraints of model minority positioning.

Recent work in behavioral economics provides additional context for understanding these identity choices. \textcite{bordaloStereotypes2016} demonstrate how stereotypes influence economic decision-making, while \textcite{bonomiIdentityBeliefsPolitical2021} show how identity affects political and economic preferences. My analysis contributes to this literature by showing how external discrimination shapes the fundamental choice of racial identity.

Taking into consideration the identity flexibility that characterizes Asian American experiences, I investigate the economic determinants driving racial self-identification decisions. \textcite{hadah2024hispanicidentity} finds that bias and self-reported Hispanic identity are negatively associated among objectively Hispanic immigrants. I aim to examine how certain personal and environmental factors influence the complexity of endogenous racial identity among Asian Americans, recognizing that the model minority stereotype creates unique economic incentive structures not present for other groups.

The empirical analysis documents how observable characteristics—individual traits and societal attitudes—affect racial identity reporting among Asian Americans. These findings have important implications for measuring racial economic disparities and understanding how discrimination operates in modern labor markets.

\section{Theoretical Framework}\label{sec:model}

I develop a theoretical framework for understanding racial identity choice that extends \textcite{akerlofEconomicsIdentity2000} to incorporate stereotype-specific costs and benefits. Unlike generic minority identification models, this framework recognizes that Asian Americans face unique utility trade-offs where racial identity can signal both positive attributes (academic achievement, work ethic) and negative characteristics (foreignness, social exclusion).

Formally, individual $i$ belongs to racial group $r_i \in \{A, W\}$, where $A$ represents Asian and $W$ represents White. Agent $i$'s utility depends on their actions and how those actions interact with their chosen racial identity $I_i$:

\begin{equation}
U_i = U_i(\pmb{a_i}, \pmb{a_{-i}}, I_i)\label{eq:util}
\end{equation}

Individual identity $I_i$ reflects personal actions, others' behaviors toward them, and societal expectations associated with their racial group:

\begin{equation}
I_i = I_i(\pmb{a_i}, \pmb{a_{-i}}; \pmb{S}_{r_{i}})\label{eq:identity}
\end{equation}

Where $\pmb{a_i}$ represents individual $i$'s actions, $\pmb{a_{-i}}$ captures others' actions affecting $i$'s identity (including anti-Asian bias), and $\pmb{S}_{r_{i}}$ denotes societal stereotypes and expectations associated with racial group membership.\footnote{This extends \textcite{akerlofEconomicsIdentity2000}'s proscription concept to encompass both negative stereotypes and positive model minority expectations.}

The key insight for Asian Americans is that $\pmb{S}_{A}$ includes both positive stereotypes (academic excellence, economic success) and negative ones (perpetual foreigner status, social exclusion). This creates context-dependent utility from Asian identification—beneficial in some settings (academic achievement contexts) but costly in others (social acceptance, political inclusion).

Individual $i$ selects actions $a_i$ to maximize utility given their racial group $r_i$, associated stereotypes $\pmb{S}_{r_{i}}$, and others' actions $\pmb{a_{-i}}$. The first-order condition becomes:

\begin{equation}
\frac{\partial U_i}{\partial a_i} + \frac{\partial U_i}{\partial I_i} \cdot \frac{d I_i}{d a_i} = 0\label{eq:foc}
\end{equation}

The solution $a_{i}^{\star}$ yields utility $U_{i}^{\star}$. Now suppose individuals can strategically choose their racial identity at cost $c$. They will switch identities when $\tilde{U_{i}}^{\star} \geq U_{i}^{\star} + c$, where $\tilde{U_{i}}^{\star}$ represents utility under the alternative racial identity.

Identity switching occurs when benefits $\tilde{U_{i}}^{\star} - U_{i}^{\star}$ exceed costs $c$. These net benefits are non-zero only when $\frac{d I_i}{d a_i} \neq 0$ and $\frac{\partial U_i}{\partial I_i} \neq 0$. This framework suggests empirical analysis should focus on: (1) individual characteristics affecting optimal actions under different racial identities, (2) contextual factors (anti-Asian bias) creating differential treatment by racial group, (3) populations with low switching costs $c$, and (4) groups whose utility significantly depends on racial identity.

From the empirical analysis, I investigate characteristics affecting individual actions under different identity choices from point (1). These characteristics include immigrant generation, mixed-race versus mono-racial family structure, etc. I also examine how anti-Asian bias influences identity choices. Finally, restricting analysis to individuals with low identity switching costs $c$ ensures the sample excludes populations unlikely to modify racial identification—for example, non-Asian Whites without Asian ancestry.

The model predicts that anti-Asian bias increases the utility differential between White and Asian identification, making identity switching more attractive. Mixed-race individuals face lower switching costs due to phenotypic ambiguity, while later-generation Asian Americans may find identity switching more feasible due to cultural assimilation.

\section{Data Sources and Measurement Strategy}\label{sec:data}

This section describes the datasets employed in the analysis. To examine relationships between social attitudes and Asian racial identity reporting, I require both subjective and objective Asian identity measures for selecting appropriate Asian American subgroups. I utilize the Integrated Public Use Microdata Series (IPUMS) Current Population Survey (CPS) \autocite{floodsarahIntegratedPublicUse2021a} with ancestry information to construct objective identity measures. I develop composite anti-Asian bias measures using \textcite{lubotskyInterpretationRegressionsMultiple2006}'s methodology to reduce attenuation bias.

\subsection{Measuring Asian Racial Identity}\label{subsec:cps}

I measure Asian racial identity using Current Population Survey (CPS) data from 2004-2021, enabling construction of objective Asian ancestry measures for minors living with parents. Following \textcite{antmanEthnicAttritionObserved2016,antmanEthnicAttritionAssimilation2020}, I utilize birthplace information for individuals, parents, and grandparents to create objective Asian ancestry indicators.\footnote{This approach parallels previous research but focuses on racial rather than ethnic categorization.} The methodology allows perfect identification of first-, second-, and third-generation Asian Americans (see Figure \ref{fig:diag} for visual representation). This approach enables construction of objective Asian ancestry measures for minors under age 17 living with parents. I also use another CPS sample of adults aged 18+ to examine whether their racial identity reporting patterns differ from minors. For the adult sample, I can only identify first- and second-generation.\footnote{Since adults still living with parents are might not representative of the overall adult population and is a rare occurrence, I do not link adults to their parents.}

The objective ancestry measure—distinct from subjective racial identification where respondents select ``Asian'' as their race—depends on birthplaces across three generations.\footnote{For this analysis, Asian countries comprise East Asian and Southeast Asian nations, including China, Hong Kong, Taiwan, Japan, Korea, Mongolia, Cambodia, Indonesia, Laos, Malaysia, Philippines, Singapore, Thailand, and Vietnam, but exclude South Asian and Middle Eastern countries, consistent with standard demographic classifications.} The three identifiable generations include: 1) first-generation immigrants born in Asian countries with both parents also born in Asian countries, 2) second-generation individuals who are US-born citizens with at least one parent born in an Asian country, 3) third-generation Asian Americans who are US-born citizens with two US-born parents and at least one grandparent born in an Asian country. It is important to note that while the ancestry measure provides an objective assessment of Asian heritage, it may not capture all nuances of racial identity. For instance, White individuals with Asian ancestry born to non-Asian parents in Asia, such as on American military bases, may not be classified as Asian in the data. To avoid potential misclassification, I remove individuals who report that they were born abroad of American parents. The sample includes Asian Americans, first-, second-, and third-generation immigrants aged 17 and younger living with parents between 2004 and 2021. Summary statistics appear in Table (\ref{tab:sumstat1}). The adult sample includes first- and second-generation Asian American immigrants aged 18 and older between 2004 and 2021. The summary statistics for the adult sample appear in Table (\ref{tab:sumstat-adults}). 

While CPS relies on household respondents (parents or caregivers) to report children's racial identity, this proxy reporting likely reflects children's actual identity since parents significantly influence identity formation. \textcite{duncanIntermarriageIntergenerationalTransmission2011} support this perspective, demonstrating no variation in Asian identification based on household respondent type. The data confirms consistent Asian identity reporting regardless of whether mother (72\%), father (72\%), or child/other caregiver (87\%) serves as respondent, as shown in Table \ref{tab:hispbyproxy}.\footnote{According to CPS guidelines, household respondents must be at least 15 years old with sufficient household knowledge. When the proxy is `self,' the respondent ranges from 15 to 17 years old.} Since my analysis compares high and low bias states, estimates remain valid provided reporting patterns don't systematically differ between these contexts.

The overall sample comprises 49\% females, with 65\% self-reporting Asian racial identity—answering affirmatively to ``what is your race.'' Average age equals 8.4 years. Approximately 52\% of mothers and 52\% of fathers hold college degrees. Additional summary statistics for the overall sample and each generation appear in Table (\ref{tab:sumstat1}).

Using parental and grandparental birthplaces, I objectively identify ethnic ancestry and categorize different family types. For second-generation children, parental birthplaces create three objective categories:
\begin{enumerate}
\item Objectively Asian-father-Asian-mother (AA)
\item Objectively Asian-father-White-mother (AW)  
\item Objectively White-father-Asian-mother (WA)
\end{enumerate}

Similarly, grandparental birthplaces create 15 objective categories for third-generation children: (1) objectively Asian paternal grandfather-Asian paternal grandmother-Asian maternal grandfather-Asian maternal grandmother (AAAA); (2) objectively White paternal grandfather-Asian paternal grandmother-Asian maternal grandfather-Asian maternal grandmother (WAAA); (3) objectively Asian paternal grandfather-White paternal grandmother-Asian maternal grandfather-Asian maternal grandmother (AWAA), etc.

My analysis employs a US population subsample; Table (\ref{tab:hispbygen}) demonstrates sufficient observations across generations. Consistent with literature on racial identity fluidity among Asian Americans, I document significant attrition among third-generation Asian Americans.\footnote{\textcite{duncanIdentifyingLaterGenerationDescendants2018,duncanSocioeconomicIntegrationImmigrant2018, antmanEthnicAttritionObserved2016,antmanEthnicAttritionAssimilation2020} document substantial identity attrition among various groups.} Furthermore, ethnic attrition patterns among adults align with those observed in children, as shown in Table (\ref{tab:hispbygen-adults}), suggesting that proxy reporting aligns with individuals' self-identification. 

\subsection{Measuring Anti-Asian Sentiment}\label{sub:lw-bias}

I construct anti-Asian sentiment measures using implicit association tests, American National Election Studies, and hate crimes targeting Asian Americans from 2004-2021. The implicit association test measures conceptual associations—for example, linking Asian Americans with negative stereotypes—and evaluative responses. Respondents rapidly categorize words into screen-displayed categories. Figure (\ref{fig:iatexamples}) shows examples from Harvard's Project Implicit skin tone test.

I employ Asian-focused implicit association test data to construct anti-Asian prejudice proxies \autocite{greenwaldMeasuringIndividualDifferences1998}. This measure has extensive social science applications, particularly in psychology. Previous research demonstrates IAT score manipulation difficulty \autocite{egloffPredictiveValidityImplicit2002}.

The IAT measures bias direction and magnitude while capturing unconscious biases individuals may be unwilling to report. Meta-analysis of over 122 IAT studies by \textcite{greenwaldMeasuringIndividualDifferences1998} finds significantly higher predictive validity for IAT compared to self-report measures. However, some research questions IAT predictive validity claims.\footnote{Research correlates IAT tests with economic outcomes \autocite{chettyRaceEconomicOpportunity2020,gloverDiscriminationSelfFulfillingProphecy2017}, voting behavior \autocite{friesePredictingVotingBehavior2007}, and health \autocite{leitnerRacialBiasAssociated2016}. IAT participation is voluntary, potentially creating selection bias. However, IAT-reflected bias serves as a prejudiced attitude proxy \autocite{chettyRaceEconomicOpportunity2020}.} Implicit Association Tests may not reliably measure or predict implicit prejudice or biased behaviors. Research shows implicit biases experience minor, temporary intervention-induced changes. Additionally, implicit bias fails to predict dictator game contributions or social pressure susceptibility, highlighting distinctions between implicit bias and biased actions \autocite{arkesAttributionsImplicitPrejudice2004,forscherMetaanalysisProceduresChange2019,leeDoesImplicitBias2018}. Therefore, I supplement IAT with explicit bias measures from American National Election Studies (ANES) to construct composite bias measures.

I develop another racial animus proxy using ANES surveys \autocite{anes2021} measuring discrimination against racial groups. ANES, conducted since 1948, enjoys widespread political science usage. The survey examines attitudes toward different racial groups, voting intentions, and political questions. I employ several 2004-2020 ANES questions measuring racial animus. The racial animus index averages responses across multiple animus-measuring questions.\footnote{Questions parallel those used by \textcite{charlesPrejudiceWagesEmpirical2008}: (1) ``Conditions Make it Difficult for Blacks to Succeed'', (2) ``Blacks Should Not Have Special Favors to Succeed'', (3) ``Blacks Must Try Harder to Succeed'', (4) ``Blacks Gotten Less than They Deserve Over the Past Few Years'', and (5) ``Feeling Thermometer Toward Asians.''} While the ANES racial animus questions primarily focus on attitudes toward Black Americans, research demonstrates that racial prejudices are highly correlated across different minority groups, with individuals who express bias toward one racial minority typically holding similar attitudes toward others \autocite{almasalkhi2023links, mora2020antiblackness}. These measures therefore capture broader patterns of racial animus that extend beyond anti-Black sentiment specifically. When combined with the Asian-focused IAT measures and hate crimes against Asian Americans in my composite index, this multi-proxy approach weights the bias measure more heavily toward Asian-specific prejudice while still capturing the general racial climate.

Finally, I incorporate Uniform Crime Reports (UCR) data quantifying hate crimes against Asian Americans \autocite{ucrbook}. Hate crime data provides tangible racially-motivated aggression and discrimination measures. Combined with implicit and explicit bias measures, this enables comprehensive prejudice understanding across states. This multidimensional approach—implicit bias, explicit bias, and hate crime statistics—offers fuller racial prejudice landscape understanding.

To reduce attenuation bias and measurement error, I follow \textcite{lubotskyInterpretationRegressionsMultiple2006} constructing composite bias measures using IAT, ANES racial animus measures, and hate crimes against Asian Americans.\footnote{I show additional methodological details appear in the Data Online Appendix, Section~\ref{sub:lw-bias}.} Figure (\ref{fig:skiniat}) graphically represents bias measures over time in most and least biased locations. Figure (\ref{fig:Asian-twostates}) shows Asian racial identity reporting in the two most and least biased locations. Lower scores indicate less bias; higher scores indicate greater racial animus. One standard deviation bias increases equivalent to moving from Washington, DC, or Vermont to North Dakota in 2020. State-level average bias over time appears in Figure (\ref{fig:skiniat-maps}) maps, with overall 2004-2021 averages in Figure (\ref{fig:iat-map-all}).

\section{From the Data: Asian Racial Identity and Attrition}\label{sec:attrition}

\subsection{Asian Racial Identity Attrition by Generation and Family Structure}

In this section, I document Asian racial identity reporting patterns and attrition across generations. First, I show that Asian racial identity attrition by generation for both children and adult samples. Second, I examine which racial identities individuals with objective Asian ancestry report (Asian, White, multiracial, etc.), presenting results for children broken down by generation and family structure (interracial versus endogamous parents).

Table (\ref{tab:hispbygen}) displays the pattern of racial attrition. Most first- and second-generation Asian Americans racially self-identify as Asian. Among first-generation Asian Americans, 96\% self-report Asian racial identity displaying weak racial attrition. Among second-generation Asian Americans, 73\% self-identify as Asian, while 31\% of third-generation Asian Americans choose Asian racial identification. Attrition among second- and third-generation Asian Americans primarily occurs among children from interracial families. Interracial second-generation Asian Americans with one Asian and one White parent report 33\% Asian identification, while those with two Asian parents report 97\% Asian identification. Among third-generation Asian Americans, 25\% report Asian identification, with 94\% of those with four Asian grandparents reporting Asian identification. Therefore, Asian racial identity attrition is substantial, driven mainly by children from interracial families.

I show the racial identity choices among adults in Table (\ref{tab:hispbygen-adults}). Since I cannot observe the birth places of grandparents among the adult sample, I show the identity choices of Asian racial identity among first- and second-generation adults. Among first-generation Asian adults, 96\% self-identify as Asian. Among second-generation native-born Asian individuals 73\%  self-identify as Asian. Breaking down by interracial families, second-generation Asian Americans with Asian parents report 95\% Asian identification, while those with one Asian and one White parent report 37\% Asian identification. Thus, the first- and second-generation Asian adult samples exhibit similar patterns to the child sample.

\subsection{Racial Identity Choices Among Those with Asian Ancestry}
I display racial identity choices among objectively Asian children by generation in Figures \ref{fig:histogram-all}-\ref{fig:histogram-thirdgen}. The figures show the distribution of racial identity choices among objectively Asian children, broken down by generation, and family structure (interracial versus endogamous parents). The figures reveal that while most first-generation Asian children identify as Asian, a significant portion of second- and third-generation children choose other racial identities, particularly those from interracial families. This highlights the complexity of racial identity formation and the influence of family structure on identity choices. Beginning in 2003, the Current Population Survey (CPS) allowed respondents to report multiple racial identities. This methodological change enabled individuals to identify with more than one racial category. In the figures, I present the racial identity choices of individuals with documented Asian ancestry, categorizing their responses into six distinct classifications: (1) Asian only, (2) White only, (3) Asian and White/Pacific Islander, (4) other non-Asian multiracial combinations, (5) Asian combined with other races, and (6) Asian/Pacific Islander.


I show the racial identity choices among all objectively Asian children in Figure \ref{fig:histogram-all}. The majority of objectively Asian children report Asian only racial identity (63\%), followed by White only (15\%) and Asian and White/Pacific Islander (15\%). Around a combined 5\% report other racial identities. 

I show the racial identity choices among first-generation Asian children in Figure \ref{fig:histogram-firstgen}. The vast majority of first-generation Asian children report Asian only racial identity (94\%), with a small portion reporting White only (3\%) and Asian and White/Pacific Islander (1\%) and approximately 1\% report other racial identities.

I show the racial identity choices among second-generation Asian children in Figure \ref{fig:histogram-secondgen} broken down by family structure. Panel (A) shows the racial identity choices among all second-generation Asian children. The majority of second-generation Asian children report Asian only racial identity (72\%), followed by Asian and White/Pacific Islander (14\%) and White only (10\%). Approximately 4\% report other racial identities. Panel (B) shows the racial identity choices among second-generation Asian children with two Asian parents, i.e. endogamous families. The vast majority of these children report Asian only racial identity (96)\%, with a small portion reporting White only (2\%) and Asian and White/Pacific Islander (1\%) and approximately 1.5\% report other racial identities. Panel (C) shows the racial identity choices among second-generation Asian children with an Asian father and a White mother, i.e. interracial families. The majority of these children report Asian only racial identity (36\%), followed by Asian and White/Pacific Islander (29\%) and White only (27\%). Approximately 8\% report other racial identities. Panel (D) shows the racial identity choices among second-generation Asian children with interracial parents---White father and an Asian mother. The majority of these children report Asian and White/Pacific Islander (40\%), followed by Asian only (29\%) and White only (24\%). Approximately 7\% report other racial identities. Therefore, racial attrition among second-generation Asian children is primarily driven by children from interracial families.

Finally, I show the racial identity choices among third-generation Asian children in Figure \ref{fig:histogram-thirdgen} broken down by family structure. Panel (A) shows the racial identity choices among all third-generation Asian children. The majority of third-generation Asian children report White only racial identity (35\%), followed by Asian only (29\%), and Asian and White/Pacific Islander (26\%). Approximately 9\% reported other other racial identities. Similar to second-generation children, racial attrition among third-generation Asian children is primarily driven by children from interracial families. Panel (B) shows the racial identity choices among third-generation Asian children with one Asian grandparent. The majority of these children report White only racial identity (54\%), followed by Asian and White/Pacific Islander (26\%) and Asian only (8\%). Approximately 12\% report other racial identities. Panel (C) shows the racial identity choices among third-generation Asian children with two Asian grandparents. The majority of these children report Asian and White/Pacific Islander (39\%), followed by White only racial identity (28\%), and Asian only (23\%). Approximately 8\% report other racial identities. Panel (D) shows the racial identity choices among third-generation Asian children with three Asian grandparents. The majority of these children report Asian only racial identity (74\%), followed by White/Pacific Islander (13\%) and White only (10\%). Approximately 3\% report other racial identities. Finally, panel (E) shows the racial identity choices among third-generation Asian children with four Asian grandparents. The vast majority of these children report Asian only racial identity (92\%), with a small portion reporting White only (4\%) and Asian and White/Pacific Islander (2\%) and approximately 1\% report other racial identities.

The patterns I presented reveal a clear racial attrition that is heterogenous by generation and family structure among children with Asian ancestry. The data demonstrate that racial identity becomes increasingly fluid across generations, with first-generation individuals overwhelmingly identifying as Asian only (94\%), while third-generation individuals are almost equally likely to identify as White only (35\%) or Asian only (29\%). Crucially, the influence of family composition becomes more pronounced in later generations, whereas endogamous Asian families maintain high rates of Asian identification across generations, interracial family structures significantly increase the likelihood of non-Asian identity choices. These findings highlight the importance of distinguishing between ancestral background and self-reported racial identity when analyzing labor market outcomes, economic mobility, and demographic trends in Asian American populations.


\section{Empirical Approach and Findings}\label{sec:empstrat}

To understand associations between Asian racial self-identification and anti-Asian bias, I estimate regressions of the following form for each generation $g$:

\begin{align}
A_{ist}^g &= \beta_1^g AntiAsianBias_{st} + \beta_2^g DadCollegeGrad_{ist} + \beta_3^g MomCollegeGrad_{ist} \nonumber \\ 
            &+ \beta_4^g Women_{ist} + X_{ist}^g\pi + \gamma_{rt} 
           + \varepsilon_{ist}; 
           \text{where } g \in \{1,2,3\} \label{eq:identity_reg_bias}
\end{align}

Where $A_{ist}^g$ represents self-reported Asian racial identity of person $i$ in state $s$ at interview time $t$, $AntiAsianBias_{st}$ represents average anti-Asian bias in state $s$ at time $t$, $DadCollegeGrad_{ist}$ and $MomCollegeGrad_{ist}$ are indicator variables equaling one if father or mother graduated college, $Women_{ist}$ indicates sex, and $X_{ist}$ represents a control vector.\footnote{Controls include quartic age, Asian population fraction in state $s$, parent types (WA, AW, or AA), grandparent types (AAAA, AAAW, etc.), and generation dummy variables.} Additionally, $\gamma_{rt}$ represents region-time fixed effects controlling for region $\times$ year specific shocks.\footnote{I exclude state fixed effects due to insufficient within-state bias variation.} Region $\times$ year controls also account for systematic regional differences in overall Asian American populations and anti-Asian bias, even with temporal variation. Throughout the analysis, I cluster standard errors at state level accounting for error term $\varepsilon_{ist}$ correlation within states over time.

Since specifications include region $\times$ year $\gamma_{rt}$, the $\beta_1^g$ coefficient summarizes individual $i$ responsiveness to anti-Asian bias changes in their residence state. In other words, $\beta_1^g$ captures associations between Asian racial identity reporting and anti-Asian bias across states within Census division regions. Additionally, $\gamma_{rt}$ fixed effects account for regional and national bias trends over time. Consequently, $\beta_1^g$ provides correlations between Asian racial identity reporting and anti-Asian bias beyond national and regional bias trends. If individuals in regional states responded similarly to bias changes, then $\beta_1^g$ equals zero.

Moreover, to further understand how explanatory variables affect Asian racial identity reporting, I estimate a multinomial logit model replacing the dependent variable $A_{ist}^g$ with categorical racial identity choices: (1) Asian only, (2) White only, (3) Asian and White/Pacific Islander. The equation is as follows:

\begin{align}
\log\left(\frac{P(Y_{ist}^g = j)}{P(Y_{ist}^g = \text{Asian only})}\right) &= \beta_{1j}^g AntiAsianBias_{st} + \beta_{2j}^g DadCollegeGrad_{ist}\nonumber \\ 
& + \beta_{3j}^g MomCollegeGrad_{ist} + \beta_{4j}^g Female_{ist} + X_{ist}^g\pi_j  \nonumber \\ 
& + \gamma_{rtj} + \varepsilon_{istj}; \quad j \in \{\text{White only}, \text{Asian and White}\} \label{eq:multinomial_logit}
\end{align}

where $Y_{ist}$ represents the categorical racial identity of person $i$ in state $s$ at interview time $t$, $j$ indexes the identity categories, with "Asian only" as the reference category, $P(Y_{ist}^g = j)$ denotes the probability of person $i$ in state $s$ at time $t$ choosing identity $j$, $\mathbf{X}_{ist}^g$ represents the vector of controls that include quartic age, Asian population fraction, parent types, grandparent types, and generation dummies, $\boldsymbol{\beta}_j^g$ denotes the coefficient vector for outcome $j$ and generation $g$, and $J$ indexes the three identity categories. Finally, $\gamma_{rtj}$ represents region-time fixed effects that would control for region $\times$ year specific shocks affecting identity choice $j$, and $\varepsilon_{istj}$ is the error term.

The model specification allows the effects of anti-Asian bias and other covariates to vary across identity choices. For instance, $\beta_{1,\text{White only}}^g$ captures how anti-Asian bias affects the log-odds of choosing "White only" versus "Asian only" identity, while $\beta_{1,\text{Asian and White}}^g$ measures the bias effect on choosing "Asian and White" versus "Asian only." This framework enables analysis of how bias, sex, and parental education differentially influences various identity strategies available to individuals with Asian ancestry.

The coefficients of interest are $\beta_{1j}^g$, $\beta_{2j}^g$, $\beta_{3j}^g$ and $\beta_{4j}^g$, which capture how anti-Asian bias, parental education, and sex influence the likelihood of selecting each identity category relative to "Asian only." I estimate separate models for each generation $g \in \{1,2,3\}$ to assess whether these relationships differ across generational status.

The multinomial specification proves particularly appropriate for analyzing Asian American identity choices because it accounts for the distinct utility individuals may derive from different identity options. Unlike binary choice models, the multinomial framework recognizes that choosing "White only" identity represents a fundamentally different strategy than selecting "Asian and White" multiracial identification, even though both alternatives to "Asian only" may respond to anti-Asian bias. This distinction is particularly important for mixed-ancestry individuals, who may strategically choose between complete ethnic distancing (reporting "White only") versus maintaining partial ethnic connection through multiracial identification ("Asian and White").

Since log odds coefficients from multinomial logit models are difficult to interpret directly, I estimate marginal effects to interpret coefficients, calculating how a one-unit change in each covariate affects the probability of selecting each identity category. For binary variables such as parental education, I calculate discrete changes in predicted probabilities when the variable shifts from zero to one, holding all other covariates at their sample means.

\section{Results}\label{sec:results}

\subsection*{Dichotomous Asian Racial Identity Reporting and Anti-Asian Bias}\label{sec:results-dichotomous}

The empirical analysis provides consistent evidence that anti-Asian bias negatively correlates with Asian racial identity reporting. These relationships are strongest among individuals with greatest identity flexibility—mixed-race individuals and later-generation Asian Americans.

I report main results from estimating equation (\ref{eq:identity_reg_bias}) in Figure (\ref{plot01-regression-gen}). I present results estimating the main specification for all generations in panel (A) and for first-, second-, and third-generation subsamples in panels (B), (C), and (D), respectively. Anti-Asian bias and Asian racial identity reporting exhibit negative associations. One standard deviation anti-Asian bias increases correlate with 9 percentage point decreases in Asian racial identity reporting. Among first- and second-generation Asian Americans, one standard deviation anti-Asian bias increases associate with 5 and 8 percentage point decreases in Asian racial identity reporting. The first-generation coefficient lacks statistical significance, but confidence intervals remain predominantly negative. Among third-generation Asian Americans, one standard deviation anti-Asian bias increases associate with 8 percentage point decreases in Asian racial identity reporting. Moreover, I find that---among all objectively Asian individuals---having a college-educated father or mother increases Asian racial identity reporting by 1 percentage point.\footnote{I show the results using county-level and MSA-level anti-Asian bias measures from the Implicit Association Test (IAT) show similar patterns. I present the county-level results in Figures \ref{plot01-regression-gen-county} and \ref{plot01-regression-byparent-county}, while I show the MSA-level in Figures \ref{plot01-regression-gen-msa} and \ref{plot01-regression-byparent-msa}.} \footnote{I present linear probability model (LPM) results alongside logit and probit estimations with marginal effects to demonstrate the consistency of findings across different functional forms. Tables \ref{regtab-all-gen}-\ref{regtab-third-gen} show that the marginal effects from logit and probit models closely align with the LPM coefficients across all generations, providing confidence in the estimated relationships between anti-Asian bias and racial identity reporting regardless of modeling approach.} Examining the effect of parental education and sex, I observe that gender has minimal effects on Asian identity reporting across most specifications. Parental education effects vary by generation: among second-generation immigrants, both maternal and paternal college education show positive associations with Asian identity reporting, while among third-generation immigrants, the effects are smaller and sometimes statistically insignificant.

I report the results using adult samples in Figure (\ref{plot01-regression-gen-adults}). I present results estimating the main specification for all adults in panel (A) and for first- and second-generation subsamples in panels (B) and (C), respectively. Anti-Asian bias and Asian racial identity reporting exhibit negative associations. One standard deviation anti-Asian bias increases correlate with 5 percentage point decreases in Asian racial identity reporting. Among first-generation Asian American adults, one standard deviation anti-Asian bias increases associate with 2 percentage point decreases in Asian racial identity reporting but is statistically insignificant. Among second-generation Asian American adults, one standard deviation anti-Asian bias increases associate with 13 percentage point decreases in Asian racial identity reporting. The adult specifications reveal additional socioeconomic influences: higher household income and years of education both positively correlate with Asian identity reporting, with each additional year of education associated with approximately 1-2 percentage point increases in the probability of reporting Asian identity.

I report identical regression results for second-generation immigrant subsamples by parent type—interracial versus endogamous parents—in Figure (\ref{plot01-regression-byparent}). I present main specification results for second-generation immigrants in panel (A) and for AA, AW, and WA children subsamples in panels (B), (C), and (D), respectively. Children from interracial families show greater bias influence. One standard deviation anti-Asian bias increases associate with 5 percentage point decreases in Asian racial identity reporting among endogamous parent children—estimates are statistically insignificant. However, one standard deviation anti-Asian bias increases associate with 15 percentage point decreases in Asian racial identity reporting among Asian father-White mother children, and 10 percentage point decreases among White father-Asian mother children. Among endogamous Asian families, maternal college education consistently shows positive effects on children's Asian identity reporting. In contrast, among mixed-race families, the effects of parental education are more heterogeneous, with college-educated mothers in Asian father-White mother families showing particularly strong positive associations with Asian identity reporting.

I report the results using adult samples in Figure (\ref{plot01-regression-byparent-adults}). I present results estimating the main specification for all second-generation adults in panel (A) and for AA, AW, and WA children subsamples in panels (B), (C), and (D), respectively. One standard deviation anti-Asian bias increases associate with 13 percentage point decreases in Asian racial identity reporting among second-generation adults. Among second-generation adults with endogamous parents, one standard deviation anti-Asian bias increases associate with 10 percentage point decreases in Asian racial identity reporting but are statistically insignificant. However, one standard deviation anti-Asian bias increases associate with a statistically insignificant 3 percentage point decreases in Asian racial identity reporting among Asian father-White mother adults, and 24 percentage point decreases among White father-Asian mother adults. Among endogamous Asian families, maternal college education consistently shows positive effects on children's Asian identity reporting. In contrast, among mixed-race families, the effects of parental education are more variable, with college-educated mothers in Asian father-White mother families showing particularly strong positive associations with Asian identity reporting.

I also report regression results for third-generation immigrant subsamples by Asian grandparent numbers in Table (\ref{regtab-bygrandparents}). Overall anti-Asian bias effects on different Asian American children types are negative but mostly statistically insignificant. One standard deviation anti-Asian bias increases associate with 69 percentage point decreases in Asian racial identity reporting among Asian American children with three Asian-born grandparents. The demographic and family characteristics reveal interesting patterns by grandparent composition. Female gender is associated with lower Asian identity reporting among those with four Asian grandparents, while parental education effects are strongest among those with two Asian grandparents, where maternal college education increases Asian identity reporting by 7 percentage points.

\subsection*{Multinomial Logit Results: Racial Identity Choices and Anti-Asian Bias} \label{sec:multinomial}

In this section, I discuss the results from the multinomial logit analysis. These results reveal how anti-Asian bias differentially affects the probability of choosing "Asian only," "White only," or "Asian and White" racial identification, with particularly pronounced effects among mixed-race Asian Americans. Given that the results are mainly driven by individuals with mixed ancestry, I focus the discussion on overall patterns and subsamples with mixed ancestry. The results from the multinomial logit models are in log-odds form, which are difficult to interpret directly. Therefore, I present predicted probabilities and marginal effects to facilitate interpretation.

\subsubsection*{Overall Patterns Among All Generations}

I report main results from estimating equation (\ref{eq:multinomial_logit}) in Figure \ref{fig:pp-all-gen}, showing predicted probabilities from the multinomial logit model, and in Figure \ref{fig:marginal-effects-all}, displaying the corresponding marginal effects.I find that the multinomial logit analysis reveals how anti-Asian bias differentially affects the probability of choosing ``Asian only,'' ``White only,'' or ``Asian and White'' racial identification across all generations. I observe that anti-Asian bias produces the most pronounced effects on racial identity choice. When anti-Asian bias increases from its lowest level (-0.7) to its highest level (1.3), I find that the predicted probability of ``Asian only'' identification decreases dramatically from 98\% to 48\%, representing a 50 percentage point decline. Simultaneously, I observe that ``White only'' identification increases substantially from 1\% to 43\%, while ``Asian and White'' identification shows a modest increase from 1\% to 8\%. The marginal effects confirm these patterns: a one-unit increase in anti-Asian bias decreases the probability of ``Asian only'' identification by approximately 14 percentage points while increasing ``White only'' identification by 11 percentage points.

In contrast, I find that gender and parental education produce minimal effects on racial identity choices across all generations. I observe that female respondents exhibit virtually identical probabilities of ``Asian only'' identification compared to males (91\% vs. 92\%), with minimal variation in ``White only'' (6\% vs. 5\%) and ``Asian and White'' (2\% vs. 3\%) identification. I find that parental education demonstrates similarly limited influence: college-educated fathers show negligible effects, with ``Asian only'' identification remaining stable at 92\% regardless of paternal education status, while ``White only'' and ``Asian and White'' probabilities remain virtually unchanged at 6\% and 2--3\% respectively. I observe that maternal college education produces minimal variation, with ``Asian only'' identification increasing only slightly from 91\% to 92\% when mothers hold college degrees. I find that the marginal effects confirm these modest associations, showing that both gender and parental college education alter ``Asian only'' identification by approximately 1 percentage point. These patterns demonstrate that I find anti-Asian bias drives racial identity choice variation across all generations, while demographic and family characteristics play secondary roles in shaping identity reporting decisions.

\subsubsection*{Second-Generation Subsample by Parental Composition}

I report main results from estimating equation (\ref{eq:multinomial_logit}) for second-generation subsamples by parental composition in Figures \ref{fig:pp-second-aw} and \ref{fig:pp-second-wa}, showing predicted probabilities for AW and WA families respectively. Figure \ref{fig:marginal-effects-second-parental} displays the corresponding marginal effects for both family types. I find that among second-generation Asian Americans, parental composition reveals interesting differences between Asian father-White mother (AW) and White father-Asian mother (WA) families. I observe that anti-Asian bias produces dramatically different effects across these family types. Among AW families, I find that anti-Asian bias effects are particularly pronounced: as bias increases from minimum to maximum levels (-0.7 to 1.3), ``Asian only'' identification probability decreases from 66\% to 28\%, representing a 38 percentage point decline, while ``White only'' identification increases from 3\% to 55\%. Among WA families, I observe even more dramatic patterns: ``Asian only'' identification decreases from 54\% to 18\%, while ``Asian and White'' identification increases from 40\% to 62\%. The marginal effects confirm these patterns: I estimate that a one-unit increase in anti-Asian bias decreases ``Asian only'' identification by approximately 13 percentage points among AW families and 15 percentage points among WA families, while increasing ``White only'' identification by 20 percentage points among AW families and 5 percentage points among WA families.

I find that gender and parental education produce markedly different effects between these family types. Among AW families, I observe that gender has minimal effects: female respondents exhibit virtually identical probabilities of ``Asian only'' identification compared to males (55\% vs. 59\%), with modest increases in ``Asian and White'' identification (32\% vs. 28\%). Among WA families, I find more pronounced gender differences: females show lower ``Asian only'' identification (29\% vs. 26\%) but substantially higher ``Asian and White'' identification (54\% vs. 62\%). When I examine the role of parental education on racial identity choices, I observe that among AW families, having a college-educated father decreases ``Asian only'' identification from 61\% to 53\%, while college-educated mothers increase it from 51\% to 62\%. Among WA families, I find that parental education has minimal effects on racial identity choices, with college-educated fathers and mothers producing virtually no change in identity probabilities across all categories.

I find that the marginal effects reveal how parental education operates differently across AW (Figure \ref{subfig:meaw}) and WA (Figure \ref{subfig:mewa})  family structures, providing insights into the mechanisms through which family composition shapes identity choices. Among AW families, I observe that parental education produces paradoxical effects: having a college-educated father decreases ``Asian only'' identification by approximately 8 percentage points while slightly decreasing ``White only'' identification by 2 percentage points, with the difference flowing toward ``Asian and White'' identification, which increases by 3 percentage points. I find that maternal college education among AW families generates stronger effects in the opposite direction, increasing ``Asian only'' identification by approximately 11 percentage points while decreasing both ``White only'' and ``Asian and White'' identification by 6 and 2 percentage points, respectively.

Among WA families, I observe different patterns where parental education effects are considerably smaller. I find that having a college-educated father produces minimal effects: decreasing ``Asian only'' identification by only 1 percentage point while slightly decreasing ``White only'' identification by 1 percentage point and increasing ``Asian and White'' identification by 4 percentage points. I observe that maternal college education among WA families shows similarly modest effects, increasing ``Asian only'' identification by 3 percentage points while having virtually no effect on ``White only'' identification and decreasing ``Asian and White'' identification by 1 percentage point.

I find that anti-Asian bias effects dwarf parental education influences in both family types. Among AW families, I find that a one-unit increase in anti-Asian bias decreases ``Asian only'' identification by 13 percentage points while dramatically increasing ``White only'' identification by 20 percentage points, with minimal effects on ``Asian and White'' identification. Among WA families, I observe that the same bias increase decreases ``Asian only'' identification by 15 percentage points but increases ``Asian and White'' identification by 8 percentage points and ``White only'' identification by only 5 percentage points. These patterns suggest that I find AW families use White identification as the primary response to bias, while WA families gravitate toward multiracial identification, with parental education serving as a moderating rather than determining factor in these strategic identity choices.

These patterns demonstrate that I find parental composition fundamentally shapes how second-generation Asian Americans respond to social bias, with WA families showing greater preference for multiracial identification and AW families displaying stronger shifts toward White identification under high bias conditions.

\subsection*{Third-Generation Subsample by Grandparent Composition}
I report main results from estimating equation (\ref{eq:multinomial_logit}) for third-generation subsamples by grandparent composition in Figures \ref{fig:pp-third-one}, \ref{fig:pp-third-two}, and \ref{fig:pp-third-three}, with corresponding marginal effects shown in Figure \ref{fig:marginal-effects-third-grandparental}. I find that third-generation Asian Americans demonstrate systematic variation in identity choice patterns based on the number of Asian grandparents, revealing how ancestral dilution affects responsiveness to social bias.

I observe that anti-Asian bias produces dramatically different effects across grandparent combinations. Among those with one Asian grandparent, I find a small response to bias: as bias increases from minimum to maximum levels (-0.7 to 1.3), ``Asian only'' identification decreases modestly from 11\% to 3\%, while ``White only'' identification decreases from 55\% to 49\%, and ``Asian and White'' increases from 34\% to 48\%. Those with two Asian grandparents show intermediate patterns: ``Asian only'' identification decreases from 48\% to 13\%, ``White only'' identification increases from 21\% to 49\%, and ``Asian and White'' identification shifts from 31\% to 38\%. Among those with three Asian grandparents, I find relatively stable effects of bias on racial identity. ``Asian only'' identification remain approximately the same at around 95\% until bias increases from 1 to 1.3 when it decreases by 20 percentage points. The same pattern persists for ``White only'' identification increases from 0\% to 21\% when bias increases from 1 to 1.3, and ``Asian and White'' identification remains constant at 4-6\% regardless of the level of bias.

I find that gender and parental education effects vary systematically across grandparent configurations. Among those with one Asian grandparent, I observe minimal gender effects: females show virtually identical patterns to males across all identity categories (6\% vs. 7\% for ``Asian only,'' 50\% vs. 56\% for ``White only,'' and 44\% vs. 37\% for ``Asian and White''). Those with two Asian grandparents show the most pronounced gender effects: females demonstrate lower ``Asian only'' identification (19\% vs. 22\%) and higher ``White only'' identification (43\% vs. 39\%). Among those with three Asian grandparents, I find modest gender differences: females exhibit slightly lower ``Asian only'' identification (93\% vs. 95\%) and higher ``Asian and White'' identification (7\% vs. 5\%). 

I find that parental education operates differently across ancestral configurations. Among those with one Asian grandparent, I observe that paternal college education decreases ``Asian only'' identification from 8\% to 5\% while increasing ``Asian and White'' identification from 38\% to 43\%. Maternal college education shows similar modest effects. Among those with two Asian grandparents, I observe the strongest parental education effects: maternal college education decreases ``Asian only'' identification from 18\% to 23\% while decreasing ``Asian and White'' identification from 45\% to 34\%. Among those with three Asian grandparents, I find that parental education produces stronger effects: paternal college education decreases ``Asian only'' identification from 97\% to 89\% while increasing ``Asian and White'' identification from 3\% to 10\%.

The marginal effects reveal that I find anti-Asian bias produces the strongest effects among those with three Asian grandparents, where a one-unit increase in bias decreases ``Asian only'' identification by approximately 2 percentage points. Among those with one or two Asian grandparents, I observe smaller and often statistically insignificant bias effects. These patterns demonstrate that I find greater Asian ancestry amplifies responsiveness to social bias, while extensive ancestral dilution reduces the salience of anti-Asian attitudes in identity choice decisions.

\section{Robustness Checks and Alternative Explanations}\label{sec:robcheck}

This section explores empirical relationships between anti-Asian bias and interracial marriages, plus migration patterns among second-generation Asian Americans as robustness checks for main analysis and proxy response effects. I examine anti-Asian bias impacts on interracial marriage likelihood, focusing on interracial couples, and Asian American migration decisions within the United States.

I investigate relationships between anti-Asian bias and interracial marriages using the following regression specification:

\begin{align}
interracial_{ist}^2 &= \beta_1^2 AntiAsianBias_{st} + X_{ist}^2\pi + \gamma_{rt} 
            + \varepsilon_{ist}  \label{eq:inter-interracial} 
\end{align}

Where $interracial_{ist}^2$ indicates interracial couples, i.e., Asian husband-White wife or White husband-Asian wife. $AntiAsianBias_{st}$ represents average anti-Asian bias in state $s$ at time $t$, and $X_{ist}^2$ represents partner-specific controls affecting marriage matching including wife's and husband's education, age, and years since US immigration.

I present equation (\ref{eq:inter-interracial}) estimation results in Table (\ref{regtab-logit-02}). One standard deviation anti-Asian bias increases raise interracial parent probabilities by 4 percentage points. Breaking down analysis by couple ethnicity: one standard deviation anti-Asian bias increases associate with 1 percentage point decreases in Asian husband-White wife marriage chances. One standard deviation anti-Asian bias increases associate with 3 percentage point increases in Asian wife-White husband chances. Anti-Asian bias and interracial marriage positive correlations may result from Asian Americans in high-bias states aiming to reduce children's Asian ethnicity signal likelihood. For example, Asian American women in high-bias states might marry non-Asian White husbands, providing children non-Asian surnames.

I also investigate relationships between anti-Asian bias and migration. Since CPS doesn't report birth states, I use 2004-2021 Censuses constructing second-generation Asian American samples \autocite{floodsarahIntegratedPublicUse2021}. I construct mover variables indicating whether second-generation Asian Americans moved from birth states to other states. I use the following models estimating relationships between anti-Asian bias and migration:

\begin{align}
BirthPlaceMigration_{ist}^2 &= \beta_1^2 AntiAsianBias_{st} 
                   + X_{ist}^2\pi + \gamma_{rt} 
                   + \varepsilon_{ist} \label{eq:migration-3} \\
BirthPlaceMigration_{ilb}^2 &= \beta_1^2 AntiAsianBias_{lb} 
                   + X_{ilb}^2\pi + \gamma_{lb} 
                   + \varepsilon_{ilb} \label{eq:migration-4}
\end{align}

Where $BirthPlaceMigration_{ist}^2$ indicates whether person $i$ in state $s$ at interview $t$ lives in states different from birth states. $BirthPlaceMigration_{ilb}^2$ indicates whether person $i$ in birthplace $l$ doesn't currently live in the same state as birth year $b$. Analysis, restricted to second-generation Asian Americans with both Asian-born parents, uses equations (\ref{eq:migration-3}) and (\ref{eq:migration-4}).

I employ two approaches defining bias variables studying relationships between bias and migration variables. First specification from equation (\ref{eq:migration-3}) estimates relationships between average bias at interview time $t$ in state $s$ and $BirthPlaceMigration_{ist}^2$. Second specification from equation (\ref{eq:migration-4}) estimates relationships between average bias in birth state $l$ at birth year $b$ and $BirthPlaceMigration_{ilb}^2$.

I also estimate whether Asian-identifying individuals tend moving from high-bias to low-bias states using:

\begin{align}
Y_{ist} &= \beta_0 + \beta_1^2 Asian_{ist} +
                   X_{ist}^2\pi
                   + \varepsilon_{ist} \label{eq:migration-5}
\end{align}

Where $Y_{ist} \equiv AntiAsianBias_{ist} - AntiAsianBias_{ilb}$, $AntiAsianBias_{ist}$ represents $i$'s anti-Asian bias in state $s$ at interview time $t$, and $AntiAsianBias_{ilb}$ represents $i$'s anti-Asian bias in birth state $l$ at birth year $b$. Analysis restricts to second-generation Asian Americans with both Asian-born parents who migrated from birth state $b$ to another state $s$.

I show the results of estimating Equations (\ref{eq:migration-3}), (\ref{eq:migration-4}), and (\ref{eq:migration-5}) in Table (\ref{regtab-mig-01}) columns (1), (2), and (3) respectively. Among second-generation immigrants, no significant correlations exist between anti-Asian bias and migration decisions. Among second-generation Asian American movers, those self-reporting Asian racial identity live in states with 0.06 standard deviations greater bias than birth states. While this result shows selection into more biased states among second-generation immigrants, it doesn't affect main results showing correlations between anti-Asian bias and Asian racial identity reporting. Since Asian-identifying individuals are movers, my bias and Asian racial identity reporting relationship assessments might underestimate bias effects.

These findings indicate negative correlations between anti-Asian bias and Asian racial identity reporting among Asian Americans. Consequently, suggesting that depending on state bias levels, racial gaps relying on self-reported identity might overestimate or underestimate discrimination effects.

Several analysis concerns exist. First, Current Population Survey (CPS) self-reported identity comes from household respondents—parents or adult caregivers. Thus, 'self-reported' racial identity might not reflect children's true identity. I view parent- or caregiver-reported identity as accurate child identity representation since parents essentially shape children's self-concepts. Also, I compare high- and low-bias states for analysis. Estimates remain unchanged if self-reporting likelihood doesn't differ between these states.

Moreover, \textcite{duncanIntermarriageIntergenerationalTransmission2011} show reported Hispanic identification doesn't vary with household respondent identity. Additionally, I present main Asian racial identity reporting effects by household respondent in Table (\ref{tab:hispbyproxy}). Main reported Asian racial identity effects equal 72 percentage points when mothers serve as proxies, 72 percentage points when fathers serve as proxies, and 87 percentage points when children or other caregivers serve as household respondents.\footnote{According to Current Population Survey (CPS) guidelines, household respondents must be at least 15 years old with sufficient household knowledge. When the proxy is `self,' respondents range from 15 to 17 years old.}. To further address this concern, I examine adult Asian American samples where individuals self-report racial identity. I find similar patterns of ethnic attrition and bias effects among adults, as shown in Table (\ref{tab:hispbygen-adults}) and Figures (\ref{plot01-regression-gen-adults}-\ref{plot01-regression-byparent-adults}).

A second concern involves IAT voluntary participation and non-representative population sampling. While I don't claim IAT anti-Asian bias proxies represent populations, \textcite{egloffPredictiveValidityImplicit2002} demonstrate manipulation difficulty. Several studies correlate IAT with economic outcomes \autocite{chettyRaceEconomicOpportunity2020,gloverDiscriminationSelfFulfillingProphecy2017}, voting behavior \autocite{friesePredictingVotingBehavior2007}, decision-making \autocite{bertrandImplicitDiscrimination2005,carlanaImplicitStereotypesEvidence2019}, and health \autocite{leitnerRacialBiasAssociated2016}. Another concern involves IAT test-taker characteristic changes over time, creating non-identical samples. I address this concern including non-parametric region $\times$ year fixed effects controlling systematic test-taker characteristic differences between regions. These changes remain controlled provided test-taker characteristic differences don't vary across states within regions. Most importantly, I use ANES racial animus measures and hate crimes against Asian Americans constructing composite bias measures reducing measurement error using \textcite{lubotskyInterpretationRegressionsMultiple2006}.

Another concern involves reverse causality between greater Asian American or Black populations in states and bias levels. Greater Asian American populations in states might affect resident bias levels. For example, more Asian Americans in California or Black Americans in Louisiana could affect California and Louisiana resident bias. To demonstrate this isn't occurring, I provide Figure (\ref{scatter-plot-1}) evidence. Figure (\ref{scatter-plot-1}) plots self-reported Asian American state percentages at specific years against average anti-Asian bias in identical states during those years. I find no correlations between anti-Asian bias and Asian American state populations, making reverse causality unlikely.

Finally, bias (prejudice) and Asian racial identity reporting relationship estimators could be biased if non-Asian-identifying individuals migrate to more prejudiced states. I've shown above this isn't occurring (Table \ref{regtab-mig-01}). I find no evidence of relationships between migration decisions and anti-Asian bias. Additionally, I find those reporting Asian racial identity moved from less biased birthplaces and lived in more biased states at survey times. Thus, my results might underestimate relationships between anti-Asian bias and Asian racial identity reporting.

\section{Conclusion}\label{sec:conc}

As American society becomes increasingly multiracial, racial identity choices will significantly influence political representation, resource allocation, and social cohesion. Understanding identity determinants are particularly important for researchers studying discrimination's role in racial economic gaps. This paper demonstrates how individual characteristics and anti-Asian sentiment influence racial identity reporting among Asian Americans.

I find that individuals with Asian ancestry are significantly less likely to racially identify as Asian in states with increases anti-Asian bias. The relationships between Asian racial identity reporting and anti-Asian bias among first-generation immigrants show one standard deviation bias increases correlating with 9 percentage point decreases in Asian racial identity reporting; results are statistical significance. Relationships between Asian racial identity reporting and anti-Asian bias are more prominent among second-generation immigrants, where one standard deviation bias increases correlate with 5 percentage point decreases in Asian racial identity reporting. Among third-generation Asian Americans, one standard deviation anti-Asian bias increases correlate with 8 percentage point decreases in Asian racial identity reporting.

Additionally, anti-Asian bias produces more substantial effects among second-generation immigrant children from mixed-race families. One standard deviation anti-Asian bias increases correlate with 15 percentage point decreases in Asian racial identity reporting among second-generation Asian American children of Asian fathers and White mothers, and 10 percentage point decreases among children of White fathers and Asian mothers. I also find anti-Asian bias positively correlates with interracial marriage and shows no correlation with migration decisions.

These results are important due to consequences for correct Asian American and minority enumeration, integration patterns, and social mobility. They indicate anti-Asian bias could significantly affect how economists estimate racial gaps. Most race and ethnicity research relies on self-reported racial and ethnic identity measures. Since anti-Asian bias negatively correlates with Asian racial identity reporting, characteristics of those avoiding Asian racial identification could produce important consequences. For example, if individuals whose identities are most likely affected by bias represent the most educated, then racial gaps will be overestimated in the most biased states. Furthermore, identity decisions likely profoundly affect people's choices, investments, and well-being.

Moreover, this study could encourage further research into relationships between bias and self-reported identities for other groups. Analysis of bias effects on self-reported identity could apply to other groups. For example, researchers could estimate bias effects on sexual minority identities and other ethnic and racial minorities such as Black, Native American, and Arab American populations in the United States. Researchers could also explore outcome differences between ethnic and racial minorities who self-report versus those who don't using restricted administrative data.

The research opens several avenues for future investigation. First, scholars could examine how recent anti-Asian violence following COVID-19 has influenced identity patterns, providing natural experiments in bias effects. Second, researchers might explore identity choices in specific institutional contexts like college admissions or workplace advancement, where model minority stereotypes create complex incentive structures. Third, analysis could extend to other Asian American subgroups, recognizing that Chinese, Korean, Vietnamese, and other communities face distinct stereotypes and discrimination patterns. Fourth, more granular geographic units, such as counties, city, or zip-code level, could provide a more precise measure of local bias. However, data limitations prevent the use of these finer geographic levels in the current analysis. Future research could explore this question using more detailed geographic identifiers if such data becomes available, allowing for a deeper understanding of how local contexts shape racial identity.

Understanding strategic racial identification among Asian Americans is essential for designing effective anti-discrimination policies and accurately measuring racial equity progress. As debates over affirmative action, immigration, and racial justice continue evolving, recognizing how Asian Americans navigate identity choices becomes increasingly critical for promoting inclusive and equitable outcomes.