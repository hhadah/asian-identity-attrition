%%%%%%%%%%%%%%%%%%%%%%%%%%%%%%%%%%%
% Main Text
%%%%%%%%%%%%%%%%%%%%%%%%%%%%%%%%%%%

\section{Introduction}\label{sec:intro}

Asian Americans represent the fastest-growing racial group in the United States, yet their experiences with discrimination and identity formation remain underexplored in economic research.\footnote{The 2020 Census counted more than 20 million Asian Americans---6.4 percent of the population—nearly double the number counted two decades earlier \autocite{floodsarahIntegratedPublicUse2021a}. The Asian American population numbers are based on the author's calculations from the Current Population Survey and US Census data.} Unlike other minority groups, Asian Americans occupy a distinctive position in America's racial hierarchy—simultaneously experiencing discrimination and being labeled as "perpetual foreigners" while being characterized through the "model minority" stereotype. This dual status creates complex incentives around racial identity choices that fundamentally differ from other groups' experiences, as Asian racial identity can signal both academic excellence and perpetual foreignness.

An extensive literature has documented Asian-White gaps in various outcomes \autocite{chiswick1983analysis, duleep2012economic, hilger2016upward, arabsheibani2010asian}, yet the role of identity selection in shaping these disparities remains understudied. The challenge lies in defining and measuring racial identity, particularly when individuals possess agency in how they racially self-identify. If reporting Asian racial identity represents a strategic choice influenced by local discrimination, measured gaps may systematically vary across geographic contexts in ways that previous research has not fully explored.

Various contextual factors, including anti-Asian sentiment and stereotype threat, can influence how individuals navigate their racial identity choices. Recent events have brought renewed attention to how external hostility shapes Asian American experiences \autocite{gover2020anti}. In this paper, I examine the determinants of Asian racial identity reporting and analyze how Asian Americans strategically select between Asian and White racial identities. Specifically, I investigate how anti-Asian bias, education, and family income shape decisions to identify racially as Asian American. I also break down the analysis by generation and family structure (interracial versus endogamous parents) and examine what racial identities individuals with objective Asian ancestry report (Asian only, White only, multiracial, etc.).

This paper has important implications for public policy and economic research for several reasons. First, if individuals respond to prejudice by avoiding Asian racial identification, conventional analyses of racial gaps may systematically underestimate disparities in the most prejudiced areas. This would lead to misunderstanding of both the extent and geographic distribution of discrimination against Asian Americans. Second, identity choices may influence measured labor market trajectories among racial groups, potentially making Asian American integration appear more successful than reality suggests, thereby reinforcing model minority stereotypes that obscure genuine barriers faced by Asian American communities. Third, strategic identity reporting affects the enumeration of Asian American populations, with implications for political representation, resource allocation, and the design of policies aimed at addressing racial inequities.

I explore how individual characteristics and societal attitudes toward Asian Americans influence racial identity reporting. I use identity and ancestry data from the Current Population Survey (CPS) combined with measures of anti-Asian bias derived from Harvard's Project Implicit Association Test (IAT), the American National Election Studies (ANES), and hate crimes targeting Asian Americans.\footnote{The IAT data comes from Harvard's Project Implicit \autocite{greenwaldMeasuringIndividualDifferences1998}. Implicit bias measures have gained prominence in economics, with IAT scores correlating with economic outcomes \autocite{chettyRaceEconomicOpportunity2020,gloverDiscriminationSelfFulfillingProphecy2017}, voting patterns \autocite{friesePredictingVotingBehavior2007}, and health disparities \autocite{leitnerRacialBiasAssociated2016}.} I ground my analysis in the theoretical framework of \textcite{akerlofEconomicsIdentity2000}, explicitly modeling how external prejudice creates differential utility from identity choices and establishing conditions under which individuals strategically modify their racial self-presentation. I also use Current Population Survey (CPS) data to study the effect of education (parental and individual) and family income on Asian racial identity reporting.

Measuring identity choices outside of lab settings is challenging, requiring both objective ancestry indicators and subjective identity measures. I leverage birthplace and ancestry data to construct objective Asian ancestry measures, then analyze deviations between objective ancestry and subjective racial identity. The analysis reveals that racial identity reporting responds to both individual characteristics and environmental factors reflecting local discrimination levels.

I document that heightened anti-Asian bias correlates with reduced Asian racial identity reporting among individuals with Asian ancestry. Specifically, a one standard deviation increase in bias corresponds to a statistically significant 9 percentage point decrease in Asian racial identification among all generations combined. When examined by generation, the effects show a 5 percentage point decrease among first-generation immigrants (statistically insignificant), an 8 percentage point decrease among second-generation individuals, and an 8 percentage point decrease among third-generation Asian Americans.

The analysis by family structure reveals additional heterogeneity: bias effects are strongest among children from mixed-race families, with a one standard deviation bias increase correlating with a 15 percentage point decrease in Asian racial identity among children of Asian fathers and White mothers, and a 10 percentage point decrease among children of White fathers and Asian mothers. Among adult samples, these patterns are even more pronounced, with second-generation adults from White father-Asian mother families showing 24 percentage point decreases in Asian racial identity in response to bias increases.

Notably, I find that more educated and wealthy Asian Americans are more likely to maintain their Asian racial identity, with college-educated parents and higher household incomes positively correlating with Asian racial identity. This selective retention of Asian identity among successful individuals creates a measurement bias: when researchers use self-reported racial identity to study Asian American outcomes, they inadvertently oversample high-achieving Asians while missing those who have strategically adopted non-Asian identities in response to bias. Consequently, studies may overestimate Asian American success and underestimate the speed of apparent assimilation, as successful Asians remain visible in the data while struggling Asians disappear into other racial categories.

Multinomial logit analyses reveal that anti-Asian bias fundamentally reshapes racial identity choices. I find that when bias increases from its lowest to highest levels, individuals dramatically shift away from ``Asian only'' racial identity, with probabilities decreasing from 98\% to 48\% across all generations, while probabilities of choosing ``White only'' racial identity rise from 1\% to 43\%. In contrast, I find that parental education produces modest and inconsistent effects that vary by family composition: maternal college education increases Asian identity reporting by approximately 11 percentage points among second-generation children of Asian father-White mother families but shows minimal effects in White father-Asian mother families, while paternal education effects remain small across both family types. These patterns suggest that anti-Asian bias, rather than socioeconomic factors, primarily drives identity choices, causing research using subjective measures to misestimate Asian-White gaps in highly prejudiced areas.

This research contributes to multiple bodies of scholarly literature in economics. First, it extends the economics of identity framework by examining how racial stereotypes---both positive and negative---influence identity choices \autocite{akerlofEconomicsIdentity2000}. Building on \textcite{charnessSocialIdentityGroup2020} and \textcite{atkinHowWeChoose2021}, I show that Asian Americans face a complex utility landscape where Asian identity can simultaneously signal competence (in educational contexts) and foreignness (in social settings). The analysis connects to stratification economics research examining how racial hierarchies shape economic outcomes \autocite{darityEconomicsIdentityOrigin2006,darityPositionPossessionsStratification2022}. This framework extends to Asian American experiences, where model minority stereotypes create unique forms of racialization distinct from other groups' experiences \autocite{goldsmithDarkLightSkin2007,hamiltonSheddingLightMarriage2009,dietteSkinShadeStratification2015}.

While \textcite{akerlofEconomicsIdentity2000} theoretical model that I use provides a logical framework for understanding how bias affects racial identity, the behavioral literature on racial identity and assimilation offers crucial empirical insights that complement this theoretical approach. Research in this tradition emphasizes how racial identity is not merely a cognitive construct but is actively performed, negotiated, and reconstructed through daily interactions and life experiences \autocite{waters1990ethnic}. \textcite{telles2008generations}'s study of Mexican Americans demonstrates how individuals strategically adapt their racial presentations across different social contexts while maintaining core identity elements across generations. 

Similarly, behavioral studies have documented how discrimination experiences shape identity salience and group attachment, with individuals developing adaptive strategies that range from ethnic distancing to reactive ethnicity depending on situational factors \autocite{zhou1997segmented}. This behavioral perspective reveals that racial identity operates as both a response to external categorization and an active process of boundary maintenance \autocite{cornell2006ethnicity}. While this literature has primarily relied on qualitative observations and ethnographic methods to document identity flexibility, the present analysis advances this understanding by quantifying these strategic choices through systematic comparison of objective ancestry measures with subjective racial identification across varying environmental contexts.

My paper also contributes to research on discrimination in economic contexts. \textcite{bertrandAreEmilyGreg2004} and \textcite{charlesPrejudiceWagesEmpirical2008} demonstrate how prejudice affects labor market outcomes, while recent work by \textcite{bursztynImmigrantNextDoor2022} explores how long-term exposure shapes attitudes. My analysis extends this literature by examining how discrimination influences the fundamental question of racial self-identification.

Within immigration and integration research, this work builds on studies examining how Asian Americans navigate assimilation processes \autocite{abramitzkyCulturalAssimilationAge2016,abramitzkyNationImmigrantsAssimilation2014}. Unlike European immigrant groups, Asian Americans face persistent ``perpetual foreigner'' stereotypes that complicate integration patterns regardless of generational status \autocite{foukaImmigrantsAmericansRace2022}. The model minority myth creates additional complexity, as Asian racial identity may carry both benefits and costs depending on context \autocite{mengIntermarriageEconomicAssimilation2005}.

This paper most closely relates to recent economic research on racial identity fluidity and strategic racial identity \autocite{hadah2024hispanicidentity, antmanEthnicAttritionObserved2016,antmanIncentivesIdentifyRacial2015,antmanAmericanIndianCasinos2021}. However, while previous work focused primarily on Hispanic ethnic attrition, Asian American identity choices operate through different economic mechanisms due to distinct stereotypes, discrimination patterns, and socioeconomic profiles. The concept of ``racial identity flexibility'' among Asian Americans reflects both the economic advantages and constraints of model minority positioning.

Recent work in behavioral economics provides additional context for understanding these identity choices. \textcite{bordaloStereotypes2016} demonstrate how stereotypes influence economic decision-making, while \textcite{bonomiIdentityBeliefsPolitical2021} show how identity affects political and economic preferences. My analysis contributes to this literature by showing how external discrimination shapes the fundamental choice of racial identity.

Recognizing the identity flexibility that characterizes Asian American experiences, I investigate the economic determinants driving racial self-identification decisions. \textcite{hadah2024hispanicidentity} finds that bias and self-reported Hispanic identity are negatively associated among objectively Hispanic immigrants. I examine how personal and environmental factors influence the complexity of endogenous racial identity among Asian Americans, recognizing that the model minority stereotype creates unique economic incentive structures not present for other groups. The empirical analysis documents how observable characteristics---individual traits and societal attitudes---affect racial identity reporting among Asian Americans. These findings have important implications for measuring racial economic disparities and understanding how discrimination operates in modern labor markets.

\section{Theoretical Framework}\label{sec:model}

I develop a theoretical framework for understanding racial identity choice that extends \textcite{akerlofEconomicsIdentity2000} to incorporate stereotype-specific costs and benefits. Unlike generic minority identification models, this framework recognizes that Asian Americans face unique utility trade-offs where racial identity can signal both positive attributes (academic achievement, work ethic) and negative characteristics (foreignness, social exclusion).

Formally, individual $i$ belongs to racial group $r_i \in \{A, W\}$, where $A$ represents Asian and $W$ represents White. Agent $i$'s utility depends on their actions and how those actions interact with their chosen racial identity $I_i$:

\begin{equation}
U_i = U_i(\pmb{a_i}, \pmb{a_{-i}}, I_i)\label{eq:util}
\end{equation}

Individual identity $I_i$ reflects personal actions, others' behaviors toward them, and societal expectations associated with their racial group:

\begin{equation}
I_i = I_i(\pmb{a_i}, \pmb{a_{-i}}; \pmb{S}_{r_{i}})\label{eq:identity}
\end{equation}

where $\pmb{a_i}$ represents individual $i$'s actions, $\pmb{a_{-i}}$ captures others' actions affecting $i$'s identity (including anti-Asian bias), and $\pmb{S}_{r_{i}}$ denotes societal stereotypes and expectations associated with racial group membership.\footnote{This extends \textcite{akerlofEconomicsIdentity2000}'s prescription concept to encompass both negative stereotypes and positive model minority expectations.}

The key insight for Asian Americans is that $\pmb{S}_{A}$ includes both positive stereotypes (academic excellence, economic success) and negative ones (perpetual foreigner status, social exclusion). This creates context-dependent utility from Asian identification---beneficial in some settings (academic achievement contexts) but costly in others (social acceptance, political inclusion).

Individual $i$ selects actions $\pmb{a_i}$ to maximize utility given their racial group $r_i$, associated stereotypes $\pmb{S}_{r_{i}}$, and others' actions $\pmb{a_{-i}}$. The first-order condition becomes:

\begin{equation}
\frac{\partial U_i}{\partial \pmb{a_i}} + \frac{\partial U_i}{\partial I_i} \cdot \frac{d I_i}{d \pmb{a_i}} = 0\label{eq:foc}
\end{equation}

The solution $a_{i}^{\star}$ yields utility $U_{i}^{\star}$. Now suppose individuals can strategically choose their racial identity at cost $c$. They will switch identities when $\tilde{U_{i}}^{\star} \geq U_{i}^{\star} + c$, where $\tilde{U_{i}}^{\star}$ represents utility under the alternative racial identity.

Identity switching occurs when benefits $\tilde{U_{i}}^{\star} - U_{i}^{\star}$ exceed costs $c$. These net benefits are non-zero only when $\frac{d I_i}{d \pmb{a_i}} \neq 0$ and $\frac{\partial U_i}{\partial I_i} \neq 0$. This framework suggests empirical analysis should focus on: (1) individual characteristics affecting optimal actions under different racial identities, (2) contextual factors (anti-Asian bias) creating differential treatment by racial group, (3) populations with low switching costs $c$, and (4) groups whose utility significantly depends on racial identity.

In the empirical analysis, I investigate characteristics affecting individual actions under different identity choices from point (1). These characteristics include immigrant generation, mixed-race versus mono-racial family structure, etc. I also examine how anti-Asian bias influences identity choices. Finally, restricting analysis to individuals with low identity switching costs $c$ ensures the sample excludes populations unlikely to modify racial identification---for example, non-Asian Whites without Asian ancestry.

The model predicts that anti-Asian bias increases the utility differential between White and Asian racial identity, making identity switching more attractive. Mixed-race individuals face lower switching costs due to phenotypic ambiguity, while later-generation Asian Americans may find identity switching more feasible due to cultural assimilation.

\section{Data Sources and Measurement Strategy}\label{sec:data}

In this section, I describe the datasets I use in the analysis. To examine relationships between social attitudes and Asian racial identity reporting, I need both subjective and objective Asian identity measures for identifying appropriate Asian American subgroups. I use the Integrated Public Use Microdata Series (IPUMS) Current Population Survey (CPS) \autocite{floodsarahIntegratedPublicUse2021a} with ancestry information through the places of birth of individuals, their parents, and grandparents to construct objective identity measures. I develop composite anti-Asian bias measures using \textcite{lubotskyInterpretationRegressionsMultiple2006}'s multiple proxy regression method to reduce attenuation bias.

\subsection{Measuring Asian Racial Identity}\label{subsec:cps}

I measure Asian racial identity using Current Population Survey (CPS) data from 2004--2021, enabling construction of objective Asian ancestry measures for minors living with parents. Following \textcite{antmanEthnicAttritionObserved2016,antmanEthnicAttritionAssimilation2020}, I utilize birthplace information for individuals, parents, and grandparents to create objective Asian ancestry indicators.\footnote{This approach parallels previous research but focuses on racial rather than ethnic categorization.} The methodology allows for the identification of first-, second-, and third-generation Asian Americans (see Figure \ref{fig:diag} for visual representation). This approach enables me to construct objective Asian ancestry measures for minors under age 17 living with parents. I also use another CPS sample of adults aged 18+ to examine whether their racial identity reporting patterns differ from minors. For the adult sample, I can only identify first- and second-generation.\footnote{Since adults still living with parents might not be representative of the overall adult population and is a rare occurrence, I do not link adults to their parents.}

The objective ancestry measure---distinct from subjective racial identification where respondents select ``Asian'' as their race---depends on birthplaces across three generations.\footnote{For this analysis, Asian countries comprise East Asian and Southeast Asian nations, including China, Hong Kong, Taiwan, Japan, Korea, Mongolia, Cambodia, Indonesia, Laos, Malaysia, Philippines, Singapore, Thailand, and Vietnam, but exclude South Asian and Middle Eastern countries, consistent with standard demographic classifications.} The three identifiable generations include: 1) first-generation immigrants born in Asian countries with both parents also born in Asian countries, 2) second-generation individuals who are US-born citizens with at least one parent born in an Asian country, and 3) third-generation Asian Americans who are US-born citizens with two US-born parents and at least one grandparent born in an Asian country. 

Note that while the ancestry measure provides an objective assessment of Asian heritage, it may not capture all nuances of racial identity. For instance, White individuals with Asian ancestry born to non-Asian parents in Asia, such as on American military bases, may be classified as Asian in the data. To avoid potential misclassification, I remove individuals who report that they were born abroad of American parents. The final sample includes Asian Americans, first-, second-, and third-generation immigrants aged 17 and younger living with parents between 2004 and 2021. I present the summary statistics in Table (\ref{tab:sumstat1}). The adult sample includes first- and second-generation Asian American immigrants aged 18 and older between 2004 and 2021. I show the summary statistics for the adult sample in Table (\ref{tab:sumstat-adults}).

While CPS relies on household respondents (parents or caregivers) to report children's racial identity, this proxy reporting likely reflects children's actual identity since parents significantly influence identity formation. \textcite{antmanEthnicAttritionAssimilation2020} supports this perspective, noting that parental reporting likely underestimates rather than overestimates ethnic attrition, as children may be more likely to drop ethnic identities once they establish separate households as adults. They also cite evidence that children's observed rates of Mexican identification do not vary systematically with which household member serves as respondent. 

My data shows consistent Asian identity reporting regardless of whether mother (72\%), father (72\%), or child/other caregiver (87\%) serves as respondent, as shown in Table \ref{tab:hispbyproxy}.\footnote{According to CPS guidelines, household respondents must be at least 15 years old with sufficient household knowledge. When the proxy is `self,' the respondent ranges from 15 to 17 years old.} Since my analysis compares high and low bias states, estimates remain valid provided reporting patterns don't systematically differ between these contexts. Furthermore, ethnic attrition patterns among adults align with those observed in children, as shown in Table (\ref{tab:hispbygen}) for children versus Table (\ref{tab:hispbygen-adults}) for adults, suggesting that proxy reporting aligns with individuals' self-identification.

The overall sample comprises 49\% females, with 65\% self-reporting Asian racial identity---answering affirmatively to ``what is your race.'' Average age is 8.4 years. Approximately 52\% of mothers and 52\% of fathers hold college degrees. Additional summary statistics for the overall sample and each generation appear in Table (\ref{tab:sumstat1}).

Using parental and grandparental birthplaces, I objectively identify ethnic ancestry and categorize different family types. For second-generation children, parental birthplaces create three objective categories:
\begin{enumerate}
\item Objectively Asian-father-Asian-mother (AA)
\item Objectively Asian-father-White-mother (AW)  
\item Objectively White-father-Asian-mother (WA)
\end{enumerate}

Similarly, grandparental birthplaces create 15 objective categories for third-generation children: (1) objectively Asian paternal grandfather-Asian paternal grandmother-Asian maternal grandfather-Asian maternal grandmother (AAAA); (2) objectively White paternal grandfather-Asian paternal grandmother-Asian maternal grandfather-Asian maternal grandmother (WAAA); (3) objectively Asian paternal grandfather-White paternal grandmother-Asian maternal grandfather-Asian maternal grandmother (AWAA), etc.

My analysis employs a subsample of the US population; Tables \ref{tab:hispbygen} and \ref{tab:sumstat-adults} demonstrate sufficient observations across generations for both adults and children. Consistent with literature on ethnic and racial identity fluidity among Asian and Hispanic Americans, I document significant attrition among third-generation Asian Americans.\footnote{\textcite{duncanIdentifyingLaterGenerationDescendants2018,duncanSocioeconomicIntegrationImmigrant2018, antmanEthnicAttritionObserved2016,antmanEthnicAttritionAssimilation2020} document substantial identity attrition among various groups.}

\subsection{Measuring Anti-Asian Sentiment}\label{sub:lw-bias}

I construct anti-Asian sentiment measures using implicit association tests, American National Election Studies, and hate crimes targeting Asian Americans from 2004--2021. The implicit association test measures conceptual associations---for example, linking Asian Americans with negative stereotypes---and evaluative responses. Respondents rapidly categorize words into screen-displayed categories. Figure (\ref{fig:iatexamples}) shows examples from Harvard's Project Implicit skin tone test.

I employ Asian-focused implicit association test data to construct anti-Asian prejudice proxies \autocite{greenwaldMeasuringIndividualDifferences1998}. This measure has extensive social science applications, particularly in psychology. Previous research demonstrates the difficulty of manipulating IAT scores \autocite{egloffPredictiveValidityImplicit2002}. The IAT measures bias direction and magnitude while capturing unconscious biases individuals may be unwilling to report. A meta-analysis of over 122 IAT studies by \textcite{greenwaldMeasuringIndividualDifferences1998} finds significantly higher predictive validity for IAT compared to self-report measures. Research correlates IAT tests with economic outcomes \autocite{chettyRaceEconomicOpportunity2020,gloverDiscriminationSelfFulfillingProphecy2017}, voting behavior \autocite{friesePredictingVotingBehavior2007}, and health \autocite{leitnerRacialBiasAssociated2016}.\footnote{IAT participation is voluntary, potentially creating selection bias. However, IAT-reflected bias serves as a proxy for prejudiced attitudes \autocite{chettyRaceEconomicOpportunity2020}.} 

However, some research questions IAT predictive validity claims. Implicit Association Tests may not reliably measure or predict implicit prejudice or biased behaviors. Research shows implicit biases experience minor, temporary intervention-induced changes. Additionally, implicit bias fails to predict dictator game contributions or social pressure susceptibility, highlighting distinctions between implicit bias and biased actions \autocite{arkesAttributionsImplicitPrejudice2004,forscherMetaanalysisProceduresChange2019,leeDoesImplicitBias2018}. Therefore, I supplement IAT with explicit bias measures from American National Election Studies (ANES) and hate crimes against Asian individuals to construct a composite bias measure.

I develop another racial animus proxy using ANES surveys \autocite{anes2021} measuring discrimination against racial groups. ANES, conducted since 1948, enjoys widespread political science usage. The survey examines attitudes toward different racial groups, voting intentions, and political questions. I employ several 2004--2020 ANES questions measuring racial animus. The racial animus index averages responses across multiple animus-measuring questions.\footnote{Questions parallel those used by \textcite{charlesPrejudiceWagesEmpirical2008}: (1) ``Conditions Make it Difficult for Blacks to Succeed'', (2) ``Blacks Should Not Have Special Favors to Succeed'', (3) ``Blacks Must Try Harder to Succeed'', (4) ``Blacks Gotten Less than They Deserve Over the Past Few Years'', and (5) ``Feeling Thermometer Toward Asians.''} While the ANES racial animus questions primarily focus on attitudes toward Black Americans, research demonstrates that racial prejudices are highly correlated across different minority groups, with individuals who express bias toward one racial minority typically holding similar attitudes toward others \autocite{almasalkhi2023links, mora2020antiblackness}. These measures therefore capture broader patterns of racial animus that extend beyond anti-Black sentiment specifically. When combined with the Asian-focused IAT measures and hate crimes against Asian Americans in my composite index, this multi-proxy approach weights the bias measure more heavily toward Asian-specific prejudice while still capturing the general racial climate.

Finally, I incorporate Uniform Crime Reports (UCR) data quantifying hate crimes against Asian Americans \autocite{ucrbook}. Hate crime data provides tangible measures of racially-motivated aggression and discrimination. Combined with implicit and explicit bias measures, this enables a comprehensive understanding of prejudice across states. This multidimensional approach---implicit bias, explicit bias, and hate crime statistics---offers a fuller understanding of the racial prejudice landscape.

To reduce attenuation bias and measurement error, I follow \textcite{lubotskyInterpretationRegressionsMultiple2006} in constructing composite bias measures using IAT, ANES racial animus measures, and hate crimes against Asian Americans.\footnote{Additional methodological details appear in the Data Online Appendix, Section~\ref{sub:lw-bias}.} Figure (\ref{fig:skiniat}) graphically represents bias measures over time in the most and least biased locations. Figure (\ref{fig:Asian-twostates}) shows Asian racial identity reporting in the two most and least biased locations. Lower scores indicate less bias; higher scores indicate greater racial animus. One standard deviation bias increase is equivalent to moving from Washington, DC, or Vermont to North Dakota in 2020. State-level average bias over time appears in Figure (\ref{fig:skiniat-maps}), with overall 2004--2021 averages in Figure (\ref{fig:iat-map-all}).

\section{From the Data: Asian Racial Identity and Attrition}\label{sec:attrition}

Table (\ref{tab:hispbygen}) displays racial attrition patterns across generations. Among first-generation Asian Americans, 96\% self-report Asian racial identity. This drops to 73\% among second-generation and 31\% among third-generation Asian Americans. Attrition is driven primarily by children from interracial families. Among second-generation children, those with two Asian parents report 97\% Asian racial identity, while those with one Asian and one White parent report only 33\%. Similarly, among third-generation children, 94\% of those with four Asian grandparents report Asian racial identity compared to 25\% overall. Adult samples show similar patterns (Table \ref{tab:hispbygen-adults}). Among second-generation adults, those with two Asian parents report 95\% Asian racial identity, while those with one Asian and one White parent report 37\%.

Figures \ref{fig:histogram-all}--\ref{fig:histogram-thirdgen} display racial identity choices among objectively Asian children by generation and family structure.\footnote{Beginning in 2003, the CPS allowed respondents to report multiple racial identities. I categorize responses into six classifications: (1) Asian only, (2) White only, (3) Asian and White/Pacific Islander, (4) other non-Asian multiracial combinations, (5) Asian combined with other races, and (6) Asian/Pacific Islander.} Among all objectively Asian children (Figure \ref{fig:histogram-all}), 63\% report Asian only identity, 15\% White only, and 15\% Asian and White/Pacific Islander. First-generation children overwhelmingly report Asian only identity (94\%, Figure \ref{fig:histogram-firstgen}). Family structure strongly predicts second-generation identity choices (Figure \ref{fig:histogram-secondgen}). Children with two Asian parents overwhelmingly report Asian only identity (96\%), while those from interracial families show substantially lower rates: 36\% for Asian father-White mother families and 29\% for White father-Asian mother families.

Third-generation patterns reveal increasing identity fluidity (Figure \ref{fig:histogram-thirdgen}). White only identity becomes most common overall (35\%), followed by Asian only (29\%). The number of Asian grandparents strongly predicts choices: those with one Asian grandparent report 54\% White only, while those with four Asian grandparents report 92\% Asian only.

These patterns demonstrate that racial identity becomes increasingly fluid across generations, with endogamous Asian families maintaining high rates of Asian racial identity while interracial family structures significantly increase non-Asian identity choices. These findings highlight the importance of distinguishing between ancestral background and self-reported racial identity when analyzing Asian American outcomes.

\section{Empirical Approach and Findings}\label{sec:empstrat}

To understand associations between Asian racial self-identification and anti-Asian bias, I estimate regressions of the following form for each generation $g$:

\begin{align}
A_{ist}^g &= \beta_1^g AntiAsianBias_{st} + \beta_2^g DadCollegeGrad_{ist} + \beta_3^g MomCollegeGrad_{ist} \nonumber \\ 
            &+ \beta_4^g Women_{ist} + X_{ist}^g\pi + \gamma_{rt} 
           + \varepsilon_{ist}; 
           \text{where } g \in \{1,2,3\} \label{eq:identity_reg_bias}
\end{align}

where $A_{ist}^g$ represents self-reported Asian racial identity of person $i$ in state $s$ at interview time $t$, $AntiAsianBias_{st}$ represents average anti-Asian bias in state $s$ at time $t$, $DadCollegeGrad_{ist}$ and $MomCollegeGrad_{ist}$ are indicator variables equaling one if father or mother graduated college, $Women_{ist}$ indicates sex, and $X_{ist}$ represents a control vector.\footnote{Controls include quartic age, Asian population fraction in state $s$, parent types (WA, AW, or AA), grandparent types (AAAA, AAAW, etc.), and generation dummy variables.} Additionally, $\gamma_{rt}$ represents region-time fixed effects controlling for region $\times$ year specific shocks.\footnote{I exclude state fixed effects due to insufficient within-state bias variation.} Region $\times$ year controls also account for systematic regional differences in overall Asian American populations and anti-Asian bias, even with temporal variation. Throughout the analysis, I cluster standard errors at the state level, accounting for correlation of the error term $\varepsilon_{ist}$ within states over time.

Since specifications include region $\times$ year fixed effects $\gamma_{rt}$, the $\beta_1^g$ coefficient summarizes individual $i$'s responsiveness to anti-Asian bias changes in their state of residence. In other words, $\beta_1^g$ captures associations between Asian racial identity reporting and anti-Asian bias across states within Census division regions. Additionally, $\gamma_{rt}$ fixed effects account for regional and national bias trends over time. Consequently, $\beta_1^g$ provides correlations between Asian racial identity reporting and anti-Asian bias beyond national and regional bias trends. If individuals in states within the same region responded similarly to bias changes, then $\beta_1^g$ would equal zero.

Moreover, to further understand how explanatory variables affect Asian racial identity reporting, I estimate a multinomial logit model replacing the depMoreover, to further understand how explanatory variables affect Asian racial identity reporting, I estimate a multinomial logit model replacing the dependent variable $A_{ist}^g$ with categorical racial identity choices: (1) Asian only, (2) White only, (3) Asian and White/Pacific Islander. The equation is as follows:

\begin{align}
\log\left(\frac{P(Y_{ist}^g = j)}{P(Y_{ist}^g = \text{Asian only})}\right) &= \beta_{1j}^g AntiAsianBias_{st} + \beta_{2j}^g DadCollegeGrad_{ist}\nonumber \\ 
& + \beta_{3j}^g MomCollegeGrad_{ist} + \beta_{4j}^g Female_{ist} + X_{ist}^g\pi_j  \nonumber \\ 
& + \gamma_{rtj} + \varepsilon_{istj}; \quad j \in \{\text{White only}, \text{Asian and White}\} \label{eq:multinomial_logit}
\end{align}

where $Y_{ist}$ represents the categorical racial identity of person $i$ in state $s$ at interview time $t$, $j$ indexes the identity categories, with ``Asian only'' as the reference category, $P(Y_{ist}^g = j)$ denotes the probability of person $i$ in state $s$ at time $t$ choosing identity $j$, $\mathbf{X}_{ist}^g$ represents the vector of controls that include quartic age, Asian population fraction, parent types, grandparent types, and generation dummies, $\boldsymbol{\beta}_j^g$ denotes the coefficient vector for outcome $j$ and generation $g$, and $J$ indexes the three identity categories. Finally, $\gamma_{rtj}$ represents region-time fixed effects that would control for region $\times$ year specific shocks affecting identity choice $j$, and $\varepsilon_{istj}$ is the error term.

The model specification allows the effects of anti-Asian bias and other covariates to vary across identity choices. For instance, $\beta_{1,\text{White only}}^g$ captures how anti-Asian bias affects the log-odds of choosing ``White only'' versus ``Asian only'' identity, while $\beta_{1,\text{Asian and White}}^g$ measures the bias effect on choosing ``Asian and White'' versus ``Asian only.'' This framework enables analysis of how bias, sex, and parental education differentially influence various identity strategies available to individuals with Asian ancestry.

The coefficients of interest are $\beta_{1j}^g$, $\beta_{2j}^g$, $\beta_{3j}^g$, and $\beta_{4j}^g$, which capture how anti-Asian bias, parental education, and sex influence the likelihood of selecting each identity category relative to ``Asian only.'' I estimate separate models for each generation $g \in \{1,2,3\}$ to assess whether these relationships differ across generational status.

The multinomial specification proves particularly appropriate for analyzing Asian American identity choices because it accounts for the distinct utility individuals may derive from different identity options. Unlike binary choice models, the multinomial framework recognizes that choosing ``White only'' identity represents a fundamentally different strategy than selecting ``Asian and White'' multiracial identity, even though both are alternatives to ``Asian only'' that may respond to anti-Asian bias. This distinction is particularly important for mixed-ancestry individuals, who may strategically choose between complete ethnic distancing (reporting ``White only'') and maintaining partial ethnic connection through multiracial identity (``Asian and White'').

Since log odds coefficients from multinomial logit models are difficult to interpret directly, I compute the predicted probabilities of each identity choice at different anti-Asian bias levels, holding other covariates constant. This approach provides more intuitive insights into how bias influences the likelihood of selecting each identity category and is easier to understand.

\subsection{Results}\label{sec:results}

\subsubsection{Dichotomous Asian Racial Identity Reporting and Anti-Asian Bias}\label{sec:results-dichotomous}

Anti-Asian bias negatively correlates with Asian racial identity reporting, with the strongest effects among mixed-race individuals and later-generation Asian Americans.

I report main results from estimating equation (\ref{eq:identity_reg_bias}) in Figure (\ref{plot01-regression-gen}), showing results for all generations (panel A) and separately by generation (panels B-D). A one standard deviation increase in anti-Asian bias correlates with a 9 percentage point decrease in Asian racial identity reporting across all generations. By generation, the effects are 5 percentage points for first-generation (statistically insignificant), 8 percentage points for second-generation, and 8 percentage points for third-generation Asian Americans. College-educated parents increase Asian racial identity reporting by approximately 1 percentage point among all objectively Asian individuals.\footnote{Results using county-level and MSA-level anti-Asian bias measures show similar patterns (Figures \ref{plot01-regression-gen-county}, \ref{plot01-regression-byparent-county}, \ref{plot01-regression-gen-msa}, and \ref{plot01-regression-byparent-msa}). Marginal effects from logit and probit models closely align with linear probability model coefficients across all generations (Tables \ref{regtab-all-gen}--\ref{regtab-third-gen}).} Gender shows minimal effects on Asian identity reporting across most specifications, while parental education effects vary by generation, with stronger positive associations among second-generation immigrants.

Adult samples show similar patterns (Figure \ref{plot01-regression-gen-adults}).\footnote{For adults with Asian ancestry, I can only observe birthplaces of the person and their parents, not grandparents, limiting analysis to first- and second-generation individuals.} A one standard deviation increase in anti-Asian bias correlates with a 5 percentage point decrease in Asian racial identity reporting across all adults, with a 2 percentage point decrease among first-generation adults (statistically insignificant) and a 13 percentage point decrease among second-generation adults. Higher household income and years of education both positively correlate with Asian identity reporting, with each additional year of education associated with approximately 1--2 percentage point increases.

Results by family structure (Figure \ref{plot01-regression-byparent}) reveal stronger bias effects among children from interracial families. Among second-generation children, a one standard deviation increase in anti-Asian bias correlates with statistically insignificant 5 percentage point decreases for those with endogamous Asian parents (panel B), but significant 15 percentage point decreases for Asian father-White mother children (panel C) and 10 percentage point decreases for White father-Asian mother children (panel D). Maternal college education consistently shows positive effects on Asian identity reporting in endogamous families and Asian father-White mother families, with particularly strong associations (approximately 11 percentage points) in the latter group.

Adult second-generation results show even larger heterogeneity by family structure (Figure \ref{plot01-regression-byparent-adults}). A one standard deviation increase in anti-Asian bias correlates with a 13 percentage point decrease across all second-generation adults, with the largest effect among White father-Asian mother adults (24 percentage points, panel D). Notably, higher education positively correlates with Asian identity reporting among mixed-race adults but shows no effect in endogamous families, suggesting that more educated individuals with Asian ancestry are more likely to maintain their Asian racial identity. This selective retention implies that studies using self-reported racial identity may overestimate Asian American success and underestimate assimilation speed, as successful Asians remain visible in the data while less successful Asians attrit to other racial categories.

Third-generation results by number of Asian grandparents (Table \ref{regtab-bygrandparents}) show mostly statistically insignificant bias effects, except among children with three Asian grandparents, where a one standard deviation increase in anti-Asian bias correlates with a 69 percentage point decrease in Asian racial identity reporting. Parental education effects are strongest among those with two Asian grandparents, where maternal college education increases Asian identity reporting by 7 percentage points.\footnote{Interaction models examining how bias effects vary by individual characteristics within mixed-race families reveal some heterogeneity (Figures \ref{fig:interaction-coefs-aw-wa} and \ref{fig:interaction-coefs-thirdgen-grandparent}). Among Asian father-White mother adults, negative effects of above-average state bias are more pronounced for those with college-educated mothers, while effects are more uniform among White father-Asian mother adults. Among third-generation individuals, interaction effects are generally small and statistically insignificant, except among those with three Asian grandparents where maternal education shows stronger moderating effects.}

\subsubsection{Multinomial Logit Results: Racial Identity Choices and Anti-Asian Bias}\label{sec:multinomial}

In this section, I discuss the results from the multinomial logit analysis. These results reveal how anti-Asian bias differentially affects the probability of choosing ``Asian only,'' ``White only,'' or ``Asian and White'' racial identity, with particularly pronounced effects among mixed-race Asian Americans. Given that the results are mainly driven by individuals with mixed ancestry, I focus the discussion on overall patterns and subsamples with mixed ancestry. The multinomial logit coefficients are in log-odds form, which are difficult to interpret directly. Therefore, I present predicted probabilities to facilitate interpretation.


\textbf{Overall Patterns Among All Generations.} Figure \ref{fig:pp-all-gen} shows predicted probabilities from the multinomial logit model across all generations. Anti-Asian bias produces the most pronounced effects on racial identity choice. When bias increases from its lowest level ($-0.7$) to its highest level ($1.3$), the predicted probability of reporting ``Asian only'' identity decreases dramatically from 98\% to 48\%, while ``White only'' identity increases from 1\% to 43\%, and ``Asian and White'' identity shows a modest increase from 1\% to 8\%. In contrast, gender and parental education produce minimal effects on racial identity choices, with male and female respondents showing virtually identical probabilities across all identity categories, and parental college education generating small, inconsistent effects.

\textbf{Second-Generation Subsample by Parental Composition.} Parental composition reveals substantial differences between Asian father-White mother (AW) and White father-Asian mother (WA) families (Figures \ref{fig:pp-second-aw} and \ref{fig:pp-second-wa}). Among AW families, as bias increases from minimum to maximum levels ($-0.7$ to $1.3$), ``Asian only'' identity decreases from 66\% to 28\%, while ``White only'' increases from 3\% to 55\%. Among WA families, ``Asian only'' decreases from 54\% to 18\%, ``Asian and White'' increases from 40\% to 62\%, and ``White only'' increases from 6\% to 20\%. Gender shows minimal effects in both family types. Parental education effects differ by family structure: among AW families, maternal college education increases ``Asian only'' identity from 51\% to 62\%, while among WA families, parental education produces virtually no change across identity categories.

Adult second-generation results (Figure \ref{fig:pp-secondgen-adults-aw-wa}) show similar patterns. Among AW families (Panel \ref{subfig:pp-secgen-aw-bias}), as bias increases from minimum to maximum levels, ``White only'' identity increases from 12\% to 39\%, while ``Asian and White'' decreases from 46\% to 16\%. Among WA families (Panel \ref{subfig:pp-secgen-wa-bias}), ``White only'' increases from 27\% to 38\%, ``Asian and White'' increases from 25\% to 40\%, and ``Asian only'' decreases from 47\% to 22\%. Gender shows minimal effects on identity choices for both family types.

\textbf{Third-Generation Subsample by Grandparent Composition.} Third-generation identity choices vary systematically with the number of Asian grandparents (Figures \ref{fig:pp-third-one}, \ref{fig:pp-third-two}, and \ref{fig:pp-third-three}). Anti-Asian bias produces dramatically different effects across ancestral compositions. Those with one Asian grandparent show modest responses to bias, with ``Asian only'' decreasing from 11\% to 3\% and ``Asian and White'' increasing from 34\% to 48\% as bias moves from minimum to maximum. Those with two Asian grandparents show stronger effects, with ``Asian only'' decreasing from 48\% to 13\% and ``White only'' increasing from 21\% to 49\%. Those with three Asian grandparents maintain high ``Asian only'' identification ($\approxeq 95\%$) until bias reaches its highest levels, when it drops 20 percentage points. 

Gender effects are minimal across all grandparent types, with males and females showing similar identity distributions. Parental education operates differently by ancestral composition: maternal college education has modest effects among those with one grandparent, strongest effects among those with two grandparents (increasing ``Asian only'' from 18\% to 23\%), and significant effects among those with three grandparents (increasing ``Asian only'' from 88\% to 96\%).

\section{Robustness Checks and Alternative Explanations}\label{sec:robcheck}

This section explores empirical relationships between anti-Asian bias and interracial marriage and migration patterns among second-generation Asian Americans as robustness checks for the main analysis and to address proxy response effects. I examine how anti-Asian bias affects interracial marriage likelihood and Asian American migration decisions within the United States.

I investigate relationships between anti-Asian bias and interracial marriages using the following regression specification:

\begin{align}
interracial_{ist}^2 &= \beta_1^2 AntiAsianBias_{st} + X_{ist}^2\pi + \gamma_{rt} 
            + \varepsilon_{ist}  \label{eq:inter-interracial} 
\end{align}

where $interracial_{ist}^2$ indicates interracial couples, i.e., Asian husband-White wife or White husband-Asian wife, $AntiAsianBias_{st}$ represents average anti-Asian bias in state $s$ at time $t$, and $X_{ist}^2$ represents partner-specific controls affecting marriage matching, including wife's and husband's education, age, and years since US immigration.

I present estimation results for equation (\ref{eq:inter-interracial}) in Table (\ref{regtab-logit-02}). A one standard deviation increase in anti-Asian bias raises interracial parent probabilities by 4 percentage points. Breaking down by couple ethnicity: a one standard deviation increase in anti-Asian bias associates with a 1 percentage point decrease in Asian husband-White wife marriage likelihood and a 3 percentage point increase in Asian wife-White husband marriage likelihood. The positive correlation between anti-Asian bias and interracial marriage may result from Asian Americans in high-bias states aiming to reduce the likelihood that their children signal Asian ethnicity. For example, Asian American women in high-bias states might marry non-Asian White husbands, providing children non-Asian surnames.

I also investigate relationships between anti-Asian bias and migration. Since CPS doesn't report birth states, I use 2004--2021 Censuses to construct second-generation Asian American samples \autocite{floodsarahIntegratedPublicUse2021}. I construct mover variables indicating whether second-generation Asian Americans moved from birth states to other states. I use the following models to estimate relationships between anti-Asian bias and migration:

\begin{align}
BirthPlaceMigration_{ist}^2 &= \beta_1^2 AntiAsianBias_{st} 
                   + X_{ist}^2\pi + \gamma_{rt} 
                   + \varepsilon_{ist} \label{eq:migration-3} \\
BirthPlaceMigration_{ilb}^2 &= \beta_1^2 AntiAsianBias_{lb} 
                   + X_{ilb}^2\pi + \gamma_{lb} 
                   + \varepsilon_{ilb} \label{eq:migration-4}
\end{align}

where $BirthPlaceMigration_{ist}^2$ indicates whether person $i$ in state $s$ at interview time $t$ lives in a state different from their birth state, and $BirthPlaceMigration_{ilb}^2$ indicates whether person $i$ born in state $l$ in year $b$ currently lives in a different state. The analysis, restricted to second-generation Asian Americans with both Asian-born parents, uses equations (\ref{eq:migration-3}) and (\ref{eq:migration-4}).

I employ two approaches to define bias variables when studying relationships between bias and migration. The first specification from equation (\ref{eq:migration-3}) estimates relationships between average bias at interview time $t$ in state $s$ and $BirthPlaceMigration_{ist}^2$. The second specification from equation (\ref{eq:migration-4}) estimates relationships between average bias in birth state $l$ at birth year $b$ and $BirthPlaceMigration_{ilb}^2$.

I also estimate whether Asian-identifying individuals tend to move from high-bias to low-bias states using:

\begin{align}
Y_{ist} &= \beta_0 + \beta_1^2 Asian_{ist} +
                   X_{ist}^2\pi
                   + \varepsilon_{ist} \label{eq:migration-5}
\end{align}

where $Y_{ist} \equiv AntiAsianBias_{ist} - AntiAsianBias_{ilb}$, $AntiAsianBias_{ist}$ represents person $i$'s anti-Asian bias in state $s$ at interview time $t$, and $AntiAsianBias_{ilb}$ represents person $i$'s anti-Asian bias in birth state $l$ at birth year $b$. The analysis restricts to second-generation Asian Americans with both Asian-born parents who migrated from birth state $l$ to another state $s$.

I show the results of estimating Equations (\ref{eq:migration-3}), (\ref{eq:migration-4}), and (\ref{eq:migration-5}) in Table (\ref{regtab-mig-01}) columns (1), (2), and (3), respectively. Among second-generation immigrants, no significant correlations exist between anti-Asian bias and migration decisions. Among second-generation Asian American movers, those self-reporting Asian racial identity live in states with 0.06 standard deviations greater bias than their birth states. While this result shows selection into more biased states among Asian-identifying second-generation immigrants, it doesn't affect the main results showing correlations between anti-Asian bias and Asian racial identity reporting. Since Asian-identifying individuals are moving to higher-bias states, my assessment of the relationship between bias and Asian racial identity reporting might underestimate bias effects.

Several concerns merit discussion. First, CPS self-reported identity comes from household respondents---parents or adult caregivers. I view parent- or caregiver-reported identity as an accurate representation of children's identity since parents essentially shape children's self-concepts. Moreover, since my analysis compares high- and low-bias states, estimates remain valid provided reporting patterns don't systematically differ between these contexts.

Moreover, \textcite{duncanIntermarriageIntergenerationalTransmission2011} show that reported Hispanic racial identity doesn't vary with household respondent identity. Consistent with this, Table (\ref{tab:hispbyproxy}) shows that Asian racial identity reporting equals 72 percentage points when mothers or fathers serve as proxies and 87 percentage points when children or other caregivers serve as household respondents.\footnote{According to CPS guidelines, household respondents must be at least 15 years old with sufficient household knowledge. When the proxy is `self,' respondents range from 15 to 17 years old.} To further address this concern, I examine adult Asian American samples where individuals self-report racial identity. I find similar patterns of ethnic attrition and bias effects among adults (Table \ref{tab:hispbygen-adults} and Figures \ref{plot01-regression-gen-adults}--\ref{plot01-regression-byparent-adults}).

Another concern involves reverse causality between larger Asian American or Black populations in states and bias levels. Greater Asian American populations might affect resident bias levels. To demonstrate this isn't occurring, Figure (\ref{scatter-plot-1}) plots self-reported Asian American state percentages against average anti-Asian bias in those states. I find no correlations between anti-Asian bias and Asian American state populations, making reverse causality unlikely.

Finally, bias and Asian racial identity reporting relationship estimates could be biased if non-Asian-identifying individuals migrate to more prejudiced states. I've shown above this isn't occurring (Table \ref{regtab-mig-01}). I find no evidence of relationships between migration decisions and anti-Asian bias. Additionally, those reporting Asian racial identity moved from less biased birthplaces and lived in more biased states at survey times. Thus, my results might underestimate relationships between anti-Asian bias and Asian racial identity reporting.

\section{Conclusion}\label{sec:conc}

As American society becomes increasingly multiracial, racial identity choices will significantly influence political representation, resource allocation, and social cohesion. Understanding the determinants of identity is particularly important for researchers studying discrimination's role in racial economic gaps. This paper demonstrates how individual characteristics and anti-Asian sentiment influence racial identity reporting among Asian Americans.

I find that individuals with Asian ancestry are significantly less likely to racially identify as Asian in states with heightened anti-Asian bias. Across all generations, a one standard deviation increase in bias correlates with a statistically significant 9 percentage point decrease in Asian racial identity reporting. When examined by generation, the relationships show a one standard deviation increase in bias correlating with a 5 percentage point decrease among first-generation immigrants (statistically insignificant), an 8 percentage point decrease among second-generation immigrants, and an 8 percentage point decrease among third-generation Asian Americans.

Anti-Asian bias produces substantially larger effects among individuals with greater identity flexibility. Among second-generation immigrant children from mixed-race families, a one standard deviation increase in anti-Asian bias correlates with a 15 percentage point decrease in Asian racial identity reporting among children of Asian fathers and White mothers, and a 10 percentage point decrease among children of White fathers and Asian mothers. Adult samples reveal even more pronounced patterns, with second-generation adults from White father-Asian mother families showing 24 percentage point decreases in response to bias increases.

Using multinomial logit analysis, I show that anti-Asian bias fundamentally reshapes racial identity. When bias increases from minimum to maximum levels, the probability of reporting ``Asian only'' identity decreases dramatically from 98\% to 48\%, while ``White only'' identity increases from 1\% to 43\%. These strategic identity choices are most pronounced among mixed-race families, where high bias environments drive substantial shifts toward White racial identity among Asian father-White mother families and toward multiracial identity among White father-Asian mother families.

These results have important consequences for Asian American enumeration, assimilation patterns, and social mobility. Since anti-Asian bias negatively correlates with Asian racial identity reporting, most race and ethnicity research relying on self-reported identity measures may systematically misestimate racial gaps. If individuals whose identities are most affected by bias are also the most educated, racial gaps will be overestimated in the most biased states. Furthermore, identity decisions likely profoundly affect people's choices, investments, and well-being.

This study encourages further research into relationships between bias and self-reported identities for other groups. Similar analysis could examine bias effects on sexual minority identities and other ethnic and racial minorities such as Black, Native American, and Arab American populations in the United States. Researchers could also explore outcome differences between ethnic and racial minorities who self-report versus those who don't using restricted administrative data.

The research opens several avenues for future investigation. First, scholars could examine how recent anti-Asian violence following COVID-19 has influenced identity patterns, providing natural experiments in bias effects. Second, researchers might explore identity choices in specific institutional contexts like college admissions or workplace advancement, where model minority stereotypes create complex incentive structures. Third, analysis could extend to other Asian American subgroups, recognizing that Chinese, Korean, Vietnamese, and other communities face distinct stereotypes and discrimination patterns. Fourth, more granular geographic analysis at the county or zip-code level could provide more precise measures of local bias, though data limitations prevent this in the current analysis.

Understanding strategic racial identification among Asian Americans is essential for designing effective anti-discrimination policies and accurately measuring racial equity progress. As debates over affirmative action, immigration, and racial justice continue to evolve, recognizing how Asian Americans navigate identity choices becomes increasingly critical for promoting inclusive and equitable outcomes.