
%%%%%%%%%%%%%%%%%%%%%%%%%%%%%%%%%%%%%%%%
% 1. Define Keywords, JEL
%%%%%%%%%%%%%%%%%%%%%%%%%%%%%%%%%%%%%%%%
\newcommand{\PAPERKEYWORDS}{\textbf{Keywords}: Economics of Minorities, Race, and Immigrants; Discrimination and Prejudice; Stratification Economics}
\newcommand{\PAPERJEL}{\textbf{JEL}: I310, J15, J71, Z13}

%%%%%%%%%%%%%%%%%%%%%%%%%%%%%%%%%%%%%%%%
% 2. Define Title
%%%%%%%%%%%%%%%%%%%%%%%%%%%%%%%%%%%%%%%%
\newcommand{\PAPERTITLE}{The Effect of Racial and Ethnic Attitudes on Asian Identity in the U.S}

%%%%%%%%%%%%%%%%%%%%%%%%%%%%%%%%%%%%%%%%
% 3. Define Authors contact information
%%%%%%%%%%%%%%%%%%%%%%%%%%%%%%%%%%%%%%%%
\newcommand{\AUTHORHADAH}{Hussain Hadah}
\newcommand{\AUTHORHADAHURL}{https://orcid.org/0000-0002-8705-6386}
\newcommand{\AUTHORHADAHINFO}{\href{\AUTHORHADAHURL}{\AUTHORHADAH}: The Murphy Institute and Department of Economics, Tulane University, Caroline Richardson Building, 62 Newcomb Place, Suite 118, New Orleans, LA 70118, United States (e-mail: \href{mailto:hhadah@tulane.edu}{hhadah@tulane.edu}, phone: +1-602-393-8077)}


%%%%%%%%%%%%%%%%%%%%%%%%%%%%%%%%%%%%%%%%
% 4. Define Thanks
%%%%%%%%%%%%%%%%%%%%%%%%%%%%%%%%%%%%%%%%
\newcommand{\ACKNOWLEDGMENTS}{
 I thank Patrick Button, Willa Friedman, Chinhui Juhn, Vikram Maheshri, and Yona Rubinstein for their support and advice. I also thank Aimee Chin, Steven Craig, German Cubas, Elaine Liu, Fan Wang, and the participants of the Applied Microeconomics Workshop at the University of Houston, the European Society for Population Economics (ESPE), the anonymous referees, and the editor for their helpful feedback.}

%%%%%%%%%%%%%%%%%%%%%%%%%%%%%%%%%%%%%%%%
% 5. Define Abstract
%%%%%%%%%%%%%%%%%%%%%%%%%%%%%%%%%%%%%%%%
\newcommand{\PAPERABSTRACT}{
I study the determinants of the choice to identify as Asian among those who could—those whose parents, grandparents, or selves were born in an Asian country. Using a multiple proxy regression approach, I construct a bias measure based on the Implicit Association Test (IAT), the American National Election Studies (ANES), and hate crimes against Asians. I find that individuals with Asian ancestry are significantly less likely to self-identify as Asian if they live in states with high levels of bias. A one standard deviation increase in bias decreases self-reported Asian identity by 9 percentage points across all generations. The effects vary by generation and family structure: bias decreases Asian identification by 5 percentage points among first-generation immigrants (statistically insignificant), 8 percentage points among second-generation, and 8 percentage points among third-generation Asian Americans. Mixed-race families show the strongest responses, with bias decreasing Asian identification by 15 percentage points among children of Asian fathers and White mothers, and 10 percentage points among children of White fathers and Asian mothers. Using multinomial analyses, I find that high bias environments drive substantial shifts from ``Asian only'' to ``White only'' identification, with probabilities changing by up to 50 percentage points. These findings have implications for the interpretation of research on racial and ethnic gaps in economic outcomes and the correct counting of the population.
\PAPERJEL}

%%%%%%%%%%%%%%%%%%%%%%%%%%%%%%%%%%%%%%%%
% 6. Define citation or availability of latest draft
%%%%%%%%%%%%%%%%%%%%%%%%%%%%%%%%%%%%%%%%
\newcommand{\PAPERDOIURL}{https://doi.org/10.1086/711654}
\newcommand{\PAPERINFO}{

 \url{\PAPERDOIURL}.
}