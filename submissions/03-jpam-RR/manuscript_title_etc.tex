
%%%%%%%%%%%%%%%%%%%%%%%%%%%%%%%%%%%%%%%%
% 1. Define Keywords, JEL
%%%%%%%%%%%%%%%%%%%%%%%%%%%%%%%%%%%%%%%%
\newcommand{\PAPERKEYWORDS}{\textbf{Keywords}: Economics of Minorities, Race, and Immigrants; Discrimination and Prejudice; Stratification Economics}
\newcommand{\PAPERJEL}{\textbf{JEL}: I310, J15, J71, Z13}

%%%%%%%%%%%%%%%%%%%%%%%%%%%%%%%%%%%%%%%%
% 2. Define Title
%%%%%%%%%%%%%%%%%%%%%%%%%%%%%%%%%%%%%%%%
\newcommand{\PAPERTITLE}{The Effect of Racial and Ethnic Attitudes on Asian Identity in the U.S}

%%%%%%%%%%%%%%%%%%%%%%%%%%%%%%%%%%%%%%%%
% 3. Define Authors contact information
%%%%%%%%%%%%%%%%%%%%%%%%%%%%%%%%%%%%%%%%
\newcommand{\AUTHORHADAH}{Hussain Hadah}
\newcommand{\AUTHORHADAHURL}{https://orcid.org/0000-0002-8705-6386}
\newcommand{\AUTHORHADAHINFO}{\href{\AUTHORHADAHURL}{\AUTHORHADAH}: The Murphy Institute and Department of Economics, Tulane University, Caroline Richardson Building, 62 Newcomb Place, Suite 118, New Orleans, LA 70118, United States (e-mail: \href{mailto:hhadah@tulane.edu}{hhadah@tulane.edu}, phone: +1-602-393-8077)}


%%%%%%%%%%%%%%%%%%%%%%%%%%%%%%%%%%%%%%%%
% 4. Define Thanks
%%%%%%%%%%%%%%%%%%%%%%%%%%%%%%%%%%%%%%%%
\newcommand{\ACKNOWLEDGMENTS}{
 I thank Patrick Button, Willa Friedman, Chinhui Juhn, Vikram Maheshri, and Yona Rubinstein for their support and advice. I also thank Aimee Chin, Steven Craig, German Cubas, Elaine Liu, Fan Wang, and the participants of the Applied Microeconomics Workshop at the University of Houston, the European Society for Population Economics (ESPE), the anonymous referees, and the editor for their helpful feedback.}

%%%%%%%%%%%%%%%%%%%%%%%%%%%%%%%%%%%%%%%%
% 5. Define Abstract
%%%%%%%%%%%%%%%%%%%%%%%%%%%%%%%%%%%%%%%%
\newcommand{\PAPERABSTRACT}{
In this paper, I study the determinants of Asian self-identification among individuals whose parents, grandparents, or selves were born in an Asian country. Using a multiple proxy regression approach, I construct a bias measure based on the Implicit Association Test (IAT), the American National Election Studies (ANES), and hate crimes against Asians. I find that individuals with Asian ancestry are significantly less likely to self-identify as Asian if they live in states with high levels of bias: a one standard deviation increase in bias decreases self-reported Asian identity by 9 percentage points across all generations, with effects of 5 percentage points for first-generation (statistically insignificant), 8 percentage points for second-generation, and 8 percentage points for third-generation Asian Americans. Children of mixed-race families show the strongest response, with bias decreasing Asian identity by 15 percentage points among children of Asian fathers and White mothers, and 10 percentage points among children of White fathers and Asian mothers. Multinomial logit results show that higher bias pushes mixed-race adults toward White or multiracial identities rather than Asian-only identification. Parental education and income have modest effects relative to bias. These findings have implications for interpreting research on racial gaps in economic outcomes and population measurement. \PAPERJEL}

%%%%%%%%%%%%%%%%%%%%%%%%%%%%%%%%%%%%%%%%
% 6. Define citation or availability of latest draft
%%%%%%%%%%%%%%%%%%%%%%%%%%%%%%%%%%%%%%%%
\newcommand{\PAPERDOIURL}{https://doi.org/10.1086/711654}
\newcommand{\PAPERINFO}{

 \url{\PAPERDOIURL}.
}
