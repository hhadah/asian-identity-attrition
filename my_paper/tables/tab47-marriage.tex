\begin{table}[H]
\centering\centering
\caption{Logistic Regression Analysis of Bias and Interracial Marriages \label{regtab-logit-03}}
\centering
\begin{threeparttable}
\begin{tabular}[t]{lccc}
\toprule
\multicolumn{2}{c}{ } & \multicolumn{1}{c}{Asian Men} & \multicolumn{1}{c}{Asian Women} \\
\cmidrule(l{3pt}r{3pt}){3-3} \cmidrule(l{3pt}r{3pt}){4-4}
  & \specialcell{(1) \\ Interracial} & \specialcell{(2) \\ Interracial} & \specialcell{(3) \\ Interracial}\\
\midrule
Bias & $1.47$*** & $0.83$ & $1.39$**\\
 & ($0.16$) & ($0.14$) & ($0.20$)\\
College Graduate: Wife & $1.42$*** & $1.55$*** & $1.75$***\\
 & ($0.06$) & ($0.09$) & ($0.09$)\\
College Graduate: Husband & $0.94$ & $0.97$ & $0.86$***\\
 & ($0.04$) & ($0.06$) & ($0.04$)\\
\midrule
Observations & $69,800$ & $52,032$ & $60,171$\\
Year $\times$ Region FE & X & X & X\\
\bottomrule
\multicolumn{4}{l}{\rule{0pt}{1em}* p $<$ 0.1, ** p $<$ 0.05, *** p $<$ 0.01}\\
\end{tabular}
\begin{tablenotes}
\small
\item[1] \footnotesize{This is the result to estimating (\ref{eq:inter-interracial}) as a
                        logistic regression. The coefficients are exponentiated, thus should be interpreted as odds ratios.}
\item[2] \footnotesize{I include controls for partners' sex, age, education, 
                      and years since immigrating to the United States.
                      Standard errors are clustered on the household level.}
\item[3] \footnotesize{Data source is the 2004-2020 Current Population Survey Data.}
\end{tablenotes}
\end{threeparttable}
\end{table}
