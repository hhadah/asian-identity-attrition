\begin{table}[H]

\caption{Relationship Between Bias and Interethnic Marriages: Marginal Effect of Ordered Logit and Probit Regressions \label{regtab-logit-01}}
\centering
\begin{threeparttable}
\begin{tabular}[t]{lcc}
\toprule
  & \specialcell{(1) \\ Probit Marginal Effect \\ at Mean} & \specialcell{(2) \\ Logistic Marginal Effect \\ at Mean}\\
\midrule
Bias & \num{-0.04}*** & \num{-0.02}***\\
 & (\num{0.00}) & (\num{0.00})\\
College Indicator: Wife & \num{-0.05}*** & \num{-0.03}***\\
 & (\num{0.00}) & (\num{0.00})\\
College Indicator: Husband & \num{-0.07}*** & \num{-0.05}***\\
 & (\num{0.00}) & (\num{0.00})\\
\bottomrule
\multicolumn{3}{l}{\rule{0pt}{1em}* p $<$ 0.1, ** p $<$ 0.05, *** p $<$ 0.01}\\
\end{tabular}
\begin{tablenotes}
\small
\item[1] \footnotesize{This is the result to estimating (\ref{eq:inter-hw}) as a
                      multinomial probit and logit regression.}
\item[2] \footnotesize{The results are of regressions where the left hand side variable is 
                      an interethnic marriage ordinal variable where a value of: 1) zero is an endogamous marriage with
                      objectively Hispanic-Husband-Hispanic-Wife; 2) one is an interethnic marriage with
                      objectively Hispanic-Husband-White-Wife; 3) two is an interethnic marriage with
                      objectively White-Husband-Hispanic-Wife. I include controls for partners' sex, age, education, 
                      and years since immigrating to the United States.
                      Standard errors are clustered on the state level.}
\item[3] \footnotesize{Data source is the 2004-2020 Current Population Survey Data.}
\end{tablenotes}
\end{threeparttable}
\end{table}
