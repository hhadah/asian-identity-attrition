\begin{table}[H]

\caption{Hispanic Self-identification by Generation \label{tab:hispbygen}}
\centering
\begin{threeparttable}
\begin{tabular}[t]{>{}lcccc}
\toprule
  & \specialcell{Self-identify \\ as Hispanic} & \specialcell{Self-identify as \\ non-Hispanic} & \specialcell{\% Self-identify \\ as Hispanic} & \specialcell{\% Self-identify \\ as non-Hispanic}\\
\midrule
\textbf{1st Gen.} & 114,657 & 5,121 & 0.96 & 0.04\\
\textbf{2nd Gen.} & 712,916 & 48,534 & 0.94 & 0.06\\
\hspace{1em}\textbf{Hispanic on:} &  &  &  \vphantom{1} & \\
\hspace{1em}\hspace{1em}\textbf{Both Sides} & 516,551 & 19,318 & 0.96 & 0.04\\
\hspace{1em}\hspace{1em}\textbf{One Side} & 196,365 & 29,216 & 0.87 & 0.13\\
\addlinespace
\textbf{3rd Gen.} & 209,206 & 45,493 & 0.82 & 0.18\\
\hspace{1em}\textbf{Hispanic on:} &  &  &  & \\
\hspace{1em}\hspace{1em}\textbf{Both Sides} & 55,401 & 2,245 & 0.96 & 0.04\\
\hspace{1em}\hspace{1em}\textbf{One Side} & 52,879 & 17,371 & 0.75 & 0.25\\
\bottomrule
\end{tabular}
\begin{tablenotes}
\item[1] The samples include children ages 17 and below who live in intact families. First-generation Hispanic immigrant children that were born in a Spanish-speaking county. Native-born second-generation Hispanic immigrant children with at least one parent born in a Spanish-speaking country. Finally, native-born third-generation Hispanic immigrant children with native-born parents and at least one grandparent born in a Spanish-speaking country.
\item[2] Data source is the 2004-2021 Current Population Survey.
\end{tablenotes}
\end{threeparttable}
\end{table}
